% !TEX encoding = UTF-8 Unicode
%%%%%%%%%%%%%%%%%%%%%%%%%%%%%%%%%%%%%%%%%%%%%%%%%%%%%%%%%%%%%%%%%%%%%%%%%%%%%%%%%%%%%%%
%%%%%%%%%%%%%%%%%%%%%%%%%%%%%%%%%%%%%%%%%%%%%%%%%%%%%%%%%%%%%%%%%%%%%%%%%%%%%%%%%%%%%%%
%%%%%%%%%%%%%                                                             %%%%%%%%%%%%%
%%%%%%%%%%%%%   Exemple de procès-verbal                                  %%%%%%%%%%%%%
%%%%%%%%%%%%%   par Pierre Tremblay, Université Laval                     %%%%%%%%%%%%%
%%%%%%%%%%%%%   2007/12/07 - version 1.3                                  %%%%%%%%%%%%%
%%%%%%%%%%%%%                                                             %%%%%%%%%%%%%
%%%%%%%%%%%%%%%%%%%%%%%%%%%%%%%%%%%%%%%%%%%%%%%%%%%%%%%%%%%%%%%%%%%%%%%%%%%%%%%%%%%%%%%
%%%%%%%%%%%%%%%%%%%%%%%%%%%%%%%%%%%%%%%%%%%%%%%%%%%%%%%%%%%%%%%%%%%%%%%%%%%%%%%%%%%%%%%
%--------------------------------------------------------------------------------------
%------------------------------------- preambule --------------------------------------
%--------------------------------------------------------------------------------------
\documentclass[12pt]{ULojpv}
%**************************************************************************************
% Chargement des packages supplementaires
%
\usepackage[utf8]{inputenc}
\usepackage{pifont}
%**************************************************************************************
%**************************************************************************************
% Definitions des parametres de l'en-tete
%
\Cours{GEL--3014 Design III }            % Nom du cours
\NumeroEquipe{05}                                     % Numero de l'equipe
\NomEquipe{}                               % Nom de l'equipe
\Objet{Procès-verbal}                                 % Nom du document
\SujetRencontre{Présentation des membres}             % Sujet de la rencontre
\DateRencontre{2012/02/26}                            % Date de la rencontre
\LocalRencontre{2012 Bibliothèque du Vachon}                            % Local de la rencontre
\HeureRencontre{08h30}                                % Heure de la rencontre
%**************************************************************************************
%--------------------------------------------------------------------------------------
%--------------------------------- corps du document ----------------------------------
%--------------------------------------------------------------------------------------
\begin{document}
\entete
\begin{enumerate}
%--------------------------------------------------------------------------------------
% nouveau point
\item \textbf{Ouverture de la réunion}

Heure: 08h30


%--------------------------------------------------------------------------------------
% nouveau point
\item \textbf{Nomination ou confirmation du président et du secrétaire}

\begin{tabular}{@{}ll}
   Président: Daniel Thibodault
   & Secrétaire: Diane Fournier
\end{tabular}


%--------------------------------------------------------------------------------------
% nouveau point
\item \textbf{Adoption de l'ordre du jour}

L'ordre du jour proposé est adopté à l'unanimité.


%--------------------------------------------------------------------------------------
% nouveau point
\item \textbf{Avancement}

\begin{enumerate}

\item Contrôleur

Asservissement des roues fonctionnel en vitesse moyenne. Reste optimisation à faire.

Problèmes à régler en communication série.

Application pour decodage d'antenne commencé avec un algorithme basé sur les timings. Devrait être testée cette semaine.

%\item Asana \#3

\item Kinect

Détection des obstacles commencée. Ajout de Francis comme ressource en support à Imane.

\item Camera

Repositionnement de la caméra sur le robot pour aider la vision du sudocube.

Contrôle de la caméra avec le pololu: pas encore commencé.

\item Assemblage du robot

Les parties majeures du robot sont assemblées. 

\item Tests unitaires

Avancés pour résolution des sudocubes. Pas encore commencés pour autres parties.

\item Résolution des sudocubes

L'algorithme permet de résoudre la plupart des cas à 4 chiffres, certains cas à 3 chiffres.

\item Pathfinding

Prototype terminé, reste à finaliser.


\item Revue des tâches à effectuer pour livrable 2 incluant délais

La rédaction n'est pas très avancée pour le moment. À venir pour cette semaine. La date de tombée pour les figures et tests est le 3 mars.

\end{enumerate}


%--------------------------------------------------------------------------------------
% nouveau point
\item \textbf{Points à traiter}

\begin{enumerate}

\item Présentation orale du 2013-02-28

Nous avons finalisé la présentation powerpoint et pratiqué la présentation orale qui aura lieu jeudi.

\end{enumerate}


%--------------------------------------------------------------------------------------
% nouveau point
\item \textbf{Divers}

\begin{enumerate}

\item Gestion hebdomadaire



\end{enumerate}

%--------------------------------------------------------------------------------------
% nouveau point
\item \textbf{Répartition des tâches}

\begin{enumerate}

\item Assignation \#1: Daniel Thibodeau et Pierre-Luc Buhler travailleront sur l'optimisation de l'asservissement des moteurs aux niveaux CAO et réel, respectivement. 

\item Assignation \#2: Tous les membres de l'équipe doivent se préparer pour la présentation orale de jeudi.

\item Assignation \#4: Daniel Thibodeau devra également compléter le registre de risques.

\item Assignation \#3: Francis et Imane doivent élaborer un algorithme de localisation des obstacles et du robot par la Kinect.

\item Assignation \#4: Pierre-Luc et Diane doivent élaborer une convention pour passer des commandes du Mac mini au microcontrôleur.

\item Assignation \#5 Diane s'occupera de la gestion de cette semaine.

\item Assignation \#6: Diane doit également compléter et tester le code de décodage Manchester sur le prototype du circuit d'antenne.

\item Assignation \#7: Philippe doit travailler sur le diagramme de séquences.

\item Assignation \#8: Imane doit terminer le plan d'intégration et la deuxième itération du diagramme de classes.

\item Assignation \#9: Tous les membres de l'équipe doivent avancer la rédaction de leur partie du rapport en vue de la remise interne du 3 mars. 

\end{enumerate}


%--------------------------------------------------------------------------------------
% nouveau point
\item \textbf{Évaluation de la réunion}

%Points individuels ou collectifs à améliorer pour accroître le rendement des réunions d'équipe.
Réunion qui s'est bien passée. Avons pratiqué la présentation orale en respectant le temps alloué de 15 minutes durant la deuxième pratique.

%--------------------------------------------------------------------------------------
% nouveau point
\item \textbf{Date, heure, lieu et objectif de la prochaine réunion}

\begin{tabular}{@{}lll}
   Date: 2013/03/05
   & Heure: 08h30
   &  Lieu: 2012 Bibliothèque du Vachon
\end{tabular}
\par
Discuter l'évolution des diffèrentes tâches distribuées à chacun des membres de l'équipe. On repassera également tous les points du livrable 2 et assignera à certaines personnes les points à compléters/modifiers.


%--------------------------------------------------------------------------------------
% nouveau point
\item \textbf{Fermeture de la réunion}

Heure: 10h30


%--------------------------------------------------------------------------------------
% nouveau point
\item \textbf{Étaient présents}

\begin{dinglist}{"33}
   \item Daniel Thibodeau
   \item Pierre-Luc Buhler
   \item Francis Valois
   \item Diane Fournier
   \item Imane Mouhtij
   \item Philippe Bourdages
	\item Olivier Sylvain
	\item Émile Arsenault
\end{dinglist}

\end{enumerate}

\end{document}
%--------------------------------------------------------------------------------------
%---------------------------------- fin du document -----------------------------------
%--------------------------------------------------------------------------------------
