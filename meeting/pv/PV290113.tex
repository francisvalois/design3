% !TEX encoding = UTF-8 Unicode
%%%%%%%%%%%%%%%%%%%%%%%%%%%%%%%%%%%%%%%%%%%%%%%%%%%%%%%%%%%%%%%%%%%%%%%%%%%%%%%%%%%%%%%
%%%%%%%%%%%%%%%%%%%%%%%%%%%%%%%%%%%%%%%%%%%%%%%%%%%%%%%%%%%%%%%%%%%%%%%%%%%%%%%%%%%%%%%
%%%%%%%%%%%%%                                                             %%%%%%%%%%%%%
%%%%%%%%%%%%%   Exemple de procès-verbal                                  %%%%%%%%%%%%%
%%%%%%%%%%%%%   par Pierre Tremblay, Université Laval                     %%%%%%%%%%%%%
%%%%%%%%%%%%%   2007/12/07 - version 1.3                                  %%%%%%%%%%%%%
%%%%%%%%%%%%%                                                             %%%%%%%%%%%%%
%%%%%%%%%%%%%%%%%%%%%%%%%%%%%%%%%%%%%%%%%%%%%%%%%%%%%%%%%%%%%%%%%%%%%%%%%%%%%%%%%%%%%%%
%%%%%%%%%%%%%%%%%%%%%%%%%%%%%%%%%%%%%%%%%%%%%%%%%%%%%%%%%%%%%%%%%%%%%%%%%%%%%%%%%%%%%%%
%--------------------------------------------------------------------------------------
%------------------------------------- preambule --------------------------------------
%--------------------------------------------------------------------------------------
\documentclass[12pt]{ULojpv}
%**************************************************************************************
% Chargement des packages supplementaires
%
\usepackage[utf8]{inputenc}
\usepackage{pifont}
%**************************************************************************************
%**************************************************************************************
% Definitions des parametres de l'en-tete
%
\Cours{GEL--3014 Design III }            % Nom du cours
\NumeroEquipe{05}                                     % Numero de l'equipe
\NomEquipe{}                               % Nom de l'equipe
\Objet{Procès-verbal}                                 % Nom du document
\SujetRencontre{Présentation des membres}             % Sujet de la rencontre
\DateRencontre{2012/01/29}                            % Date de la rencontre
\LocalRencontre{2012 Bibliothèque du Vachon}                            % Local de la rencontre
\HeureRencontre{08h30}                                % Heure de la rencontre
%**************************************************************************************
%--------------------------------------------------------------------------------------
%--------------------------------- corps du document ----------------------------------
%--------------------------------------------------------------------------------------
\begin{document}
\entete
\begin{enumerate}
%--------------------------------------------------------------------------------------
% nouveau point
\item \textbf{Ouverture de la réunion}

Heure: 08h30


%--------------------------------------------------------------------------------------
% nouveau point
\item \textbf{Nomination ou confirmation du président et du secrétaire}

\begin{tabular}{@{}ll}
   Président: Daniel Thibodault
   & Secrétaire: Pierre-Luc Buhler
\end{tabular}


%--------------------------------------------------------------------------------------
% nouveau point
\item \textbf{Adoption de l'ordre du jour}

L'ordre du jour proposé est adopté à l'unanimité.


%--------------------------------------------------------------------------------------
% nouveau point
\item \textbf{Affaires découlant de la dernière réunion}

\begin{enumerate}

\item Mettre en place mécanisme de l'organisation de l'équipe

\begin{enumerate}

\item Flash Meeting

Tout les membre se sont entendu pour utiliser la méthode proposer par Daniel pour que les réunions soient courtes et complètes.




%\item Asana \#3


\end{enumerate}

\item Confirmer choix de technologies

\begin{enumerate}
\item Traitement de texte

Tous les membres se sont entendus pour utiliser Latex comme outils de traitement de texte.

\item GitHub

Tous les membres ont installé GitHub et on accepter de l'utiliser pour la gestion des fichiers/historiques.

\item Circuits

Les GEL ont décidés d'utiliser Altium comme outils de simulation et KiCad comme outils pour faire les plans des circuits.

\item Diagramme , graphique, etc.

Tous ont accpeté d'utiliser Lucid Chart pour la création de digrammet, tableau, etc.

\item Logiciel de gestion de projet

L'équipe à choisie d'utiliser GanttProject comme logiciel de gestion de projet plutôt que MSProject.

\end{enumerate}

\item Retour sur l'énoncé global du projet

Tous les membres ont discuté du projet dans son ensemble lors de la division des tâches.

\item Revue des tâches à effectuer pour livrable 1 incluant délais

Les membres transmetteront les différents modifications qu'ils veulent apporter au différentes parties effectuées depuis la dernière réunion. Les parties effecutées sont:
\begin{itemize}
\item (Diane) diagramme de "use case".
\item (Daniel et Francis) DPF et digramme de contexte.
\item (Imane) Digramme de classe.

\end{itemize}

\end{enumerate}


%--------------------------------------------------------------------------------------
% nouveau point
\item \textbf{Points à traiter}

\begin{enumerate}

\item Prototype

Nous avons discuté des différents prototypes dont la descriptions est exigée pour le 1er livrable. Principalement, lors de la divison des tâches.

\end{enumerate}


%--------------------------------------------------------------------------------------
% nouveau point
\item \textbf{Divers}

\begin{enumerate}

\item Gestion hebdomadaire

Nous avons décidé qu'à chaque semaine un membre de l'équipe différent fera le procès verbale de la dernière réunion, ainsi qu'un fichier Excel de gestion hebdomadaire (tableau division des tâches/nombre d'heures par tâche/problèmes/actions effectuées/etc).

\end{enumerate}

%--------------------------------------------------------------------------------------
% nouveau point
\item \textbf{Répartition des tâches}

\begin{enumerate}

\item Assignation \#1: Daniel Thibodeau, Émile Arsenault, et Francis Valois doivent tester les moteurs fournies et travailleront sur l'élaboration des circuits d'alimentations

\item Assignation \#2: Émile Arsenault doit effectuer des recherches pour trouver une batterie qui sera utilisée pour alimenter le robot.

\item Assignation \#4: Émile Arsenault devra également faire le schéma du décodeur d'antenne.

\item Assignation \#3: Daniel doit élaborer un système de préhension du crayon.

\item Assignation \#4: Pierre-Luc et Diane doivent choisir un microcontroleur et travailler sur l'écran LCD utilisé pour l'affichage.

\item Assignation \#5 Pierre-Luc s'occupera de la gestion de cette semaine.

\item Assignation \#6: Imane doit commencer à travailler sur le fonctionnement de la Kinect.

\item Assignation \#7: Imane et Olivier doivent commencer à travailler sur le système de pathfinding du robot.

\item Assignation \#8: Olivier doit trouver un algorithme permettant de trouver la solution d'un sudocube.

\item Assignation \#7: Philippe doit préparer le ROS pour le 1er livrable et aidera ses collègues.

\end{enumerate}


%--------------------------------------------------------------------------------------
% nouveau point
\item \textbf{Évaluation de la réunion}

%Points individuels ou collectifs à améliorer pour accroître le rendement des réunions d'équipe.
La réunion c'est très bien déroulé. Beaucoup de points ont pu être abordés à l'intérieur de peu de temps.

%--------------------------------------------------------------------------------------
% nouveau point
\item \textbf{Date, heure, lieu et objectif de la prochaine réunion}

\begin{tabular}{@{}lll}
   Date: 2013/02/05
   & Heure: 08h30
   &  Lieu: 2012 Bibliothèque du Vachon
\end{tabular}
\par
Discuter l'évolution des diffèrentes tâches distribuées à chacun des membres de l'équipe. On repassera également tous les points du livrable 1 et assignera à certaines personnes les points à compléters/modifiers. Nous devrons aussi discuter de Asana pusiqu'il a été oublié lors de la réunion.


%--------------------------------------------------------------------------------------
% nouveau point
\item \textbf{Fermeture de la réunion}

Heure: 09h40


%--------------------------------------------------------------------------------------
% nouveau point
\item \textbf{Étaient présents}

\begin{dinglist}{"33}
   \item Daniel Thibodeau
   \item Pierre-Luc Buhler
   \item Francis Valois
   \item Diane Fournier
   \item Imane Mouhtij
   \item Philippe Bourdages
	\item Olivier Sylvain
	\item Émile Arsenault
\end{dinglist}

\end{enumerate}

\end{document}
%--------------------------------------------------------------------------------------
%---------------------------------- fin du document -----------------------------------
%--------------------------------------------------------------------------------------
