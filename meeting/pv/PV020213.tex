% !TEX encoding = UTF-8 Unicode
%%%%%%%%%%%%%%%%%%%%%%%%%%%%%%%%%%%%%%%%%%%%%%%%%%%%%%%%%%%%%%%%%%%%%%%%%%%%%%%%%%%%%%%
%%%%%%%%%%%%%%%%%%%%%%%%%%%%%%%%%%%%%%%%%%%%%%%%%%%%%%%%%%%%%%%%%%%%%%%%%%%%%%%%%%%%%%%
%%%%%%%%%%%%%                                                             %%%%%%%%%%%%%
%%%%%%%%%%%%%   Exemple de procès-verbal                                  %%%%%%%%%%%%%
%%%%%%%%%%%%%   par Pierre Tremblay, Université Laval                     %%%%%%%%%%%%%
%%%%%%%%%%%%%   2007/12/07 - version 1.3                                  %%%%%%%%%%%%%
%%%%%%%%%%%%%                                                             %%%%%%%%%%%%%
%%%%%%%%%%%%%%%%%%%%%%%%%%%%%%%%%%%%%%%%%%%%%%%%%%%%%%%%%%%%%%%%%%%%%%%%%%%%%%%%%%%%%%%
%%%%%%%%%%%%%%%%%%%%%%%%%%%%%%%%%%%%%%%%%%%%%%%%%%%%%%%%%%%%%%%%%%%%%%%%%%%%%%%%%%%%%%%
%--------------------------------------------------------------------------------------
%------------------------------------- preambule --------------------------------------
%--------------------------------------------------------------------------------------
\documentclass[12pt]{ULojpv}
%**************************************************************************************
% Chargement des packages supplementaires
%
\usepackage[utf8]{inputenc}
\usepackage{pifont}
%**************************************************************************************
%**************************************************************************************
% Definitions des parametres de l'en-tete
%
\Cours{GEL--3014 Design III }            % Nom du cours
\NumeroEquipe{05}                                     % Numero de l'equipe
\NomEquipe{les pierres à feu}                               % Nom de l'equipe
\Objet{Procès-verbal}                                 % Nom du document
\SujetRencontre{Planification livrable 1}             % Sujet de la rencontre
\DateRencontre{2013/02/05}                            % Date de la rencontre
\LocalRencontre{bibliothèque}                            % Local de la rencontre
\HeureRencontre{08h30}                                % Heure de la rencontre
%**************************************************************************************
%--------------------------------------------------------------------------------------
%--------------------------------- corps du document ----------------------------------
%--------------------------------------------------------------------------------------
\begin{document}
\entete
\begin{enumerate}
%--------------------------------------------------------------------------------------
% nouveau point
\item \textbf{Ouverture de la réunion}

Heure: 08h30


%--------------------------------------------------------------------------------------
% nouveau point
\item \textbf{Nomination ou confirmation du président et du secrétaire}

\begin{tabular}{@{}ll}
   Président: Daniel Thibodeau
   & Secrétaire: Émile Arsenault
\end{tabular}


%--------------------------------------------------------------------------------------
% nouveau point
\item \textbf{Adoption de l'ordre du jour}

L'ordre du jour proposé est adopté à l'unanimité.


%--------------------------------------------------------------------------------------
% nouveau point
\item \textbf{Lecture et adoption du procès-verbal de la réunion du 25 décembre 2007}

Le procès-verbal proposé est adopté à l'unanimité.


%--------------------------------------------------------------------------------------
% nouveau point
\item \textbf{Affaires découlant du procès-verbal}

\begin{enumerate}

\item Livrable 1 \#1

le livrable 1 devra être finalisé vendredi le 8 février afin de se garder une marge de manœuvre en cas de retard.

\begin{enumerate}

\item Mise en page des tableau \#1

Francis s'occupera de la mise en page des tableaux sur LaTeX. 


\item Description des prototypes  \#2

Chacun des membres de l'équipe écrira le ou les textes de ou des prototypes avec le ou lesquels il est le plus à l'aise.

\end{enumerate}

\item Poursuite des tâche \#2

Parallèlement à la rédaction du livrable 1, chacun des membres de l'équipe devra poursuivre ses avancements dans chacune des parties du robot.

\end{enumerate}


%--------------------------------------------------------------------------------------
% nouveau point



%--------------------------------------------------------------------------------------
% nouveau point
\item \textbf{Divers}

Aucun élément à discuter.


%--------------------------------------------------------------------------------------
% nouveau point

%--------------------------------------------------------------------------------------
% nouveau point
\item \textbf{Évaluation de la réunion}

Garder comme objectif des réunions concises qui ne s'écartent pas de l'objectif de départ.

%--------------------------------------------------------------------------------------
% nouveau point
\item \textbf{Date, heure, lieu et objectif de la prochaine réunion}

\begin{tabular}{@{}lll}
   Date: 2013/02/12
   & Heure: 08h30
   &  Lieu: Bibliothèque VCH
\end{tabular}
\par
Description de l'objectif ou des objectifs.

%--------------------------------------------------------------------------------------
% nouveau point
\item \textbf{Fermeture de la réunion}

Heure: 10h42


%--------------------------------------------------------------------------------------
% nouveau point
\item \textbf{Étaient présents}

\begin{dinglist}{"33}
   \item Daniel Thibodeau
   \item Pierre-Luc Buhler
   \item Francis Valois
   \item Diane Fournier
   \item Imane Mouhtij
   \item Philippe Bourdages
	\item Olivier Sylvain
	\item Émile Arsenault
\end{dinglist}

\end{enumerate}

\end{document}
%--------------------------------------------------------------------------------------
%---------------------------------- fin du document -----------------------------------
%--------------------------------------------------------------------------------------
