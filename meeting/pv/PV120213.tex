% !TEX encoding = UTF-8 Unicode
%%%%%%%%%%%%%%%%%%%%%%%%%%%%%%%%%%%%%%%%%%%%%%%%%%%%%%%%%%%%%%%%%%%%%%%%%%%%%%%%%%%%%%%
%%%%%%%%%%%%%%%%%%%%%%%%%%%%%%%%%%%%%%%%%%%%%%%%%%%%%%%%%%%%%%%%%%%%%%%%%%%%%%%%%%%%%%%
%%%%%%%%%%%%%                                                             %%%%%%%%%%%%%
%%%%%%%%%%%%%   Exemple de procès-verbal                                  %%%%%%%%%%%%%
%%%%%%%%%%%%%   par Pierre Tremblay, Université Laval                     %%%%%%%%%%%%%
%%%%%%%%%%%%%   2007/12/07 - version 1.3                                  %%%%%%%%%%%%%
%%%%%%%%%%%%%                                                             %%%%%%%%%%%%%
%%%%%%%%%%%%%%%%%%%%%%%%%%%%%%%%%%%%%%%%%%%%%%%%%%%%%%%%%%%%%%%%%%%%%%%%%%%%%%%%%%%%%%%
%%%%%%%%%%%%%%%%%%%%%%%%%%%%%%%%%%%%%%%%%%%%%%%%%%%%%%%%%%%%%%%%%%%%%%%%%%%%%%%%%%%%%%%
%--------------------------------------------------------------------------------------
%------------------------------------- preambule --------------------------------------
%--------------------------------------------------------------------------------------
\documentclass[12pt]{ULojpv}
%**************************************************************************************
% Chargement des packages supplementaires
%
\usepackage[utf8]{inputenc}
\usepackage{pifont}
%**************************************************************************************
%**************************************************************************************
% Definitions des parametres de l'en-tete
%
\Cours{GEL--3014 Design III }            % Nom du cours
\NumeroEquipe{05}                                     % Numero de l'equipe
\NomEquipe{les pierres à feu}                               % Nom de l'equipe
\Objet{Procès-verbal}                                 % Nom du document
\SujetRencontre{Répartition des tâches livrable 2}             % Sujet de la rencontre
\DateRencontre{2013/02/12}                            % Date de la rencontre
\LocalRencontre{Cafétéria}                            % Local de la rencontre
\HeureRencontre{08h45}                                % Heure de la rencontre
%**************************************************************************************
%--------------------------------------------------------------------------------------
%--------------------------------- corps du document ----------------------------------
%--------------------------------------------------------------------------------------
\begin{document}
\entete
\begin{enumerate}
%--------------------------------------------------------------------------------------
% nouveau point
\item \textbf{Ouverture de la réunion}

Heure: 08h45


%--------------------------------------------------------------------------------------
% nouveau point
\item \textbf{Nomination ou confirmation du président et du secrétaire}

\begin{tabular}{@{}ll}
   Président: Daniel Thibodeau
   & Secrétaire: Imane Mouhtij
\end{tabular}


%--------------------------------------------------------------------------------------
% nouveau point
\item \textbf{Adoption de l'ordre du jour}

L'ordre du jour proposé est adopté à l'unanimité.


%--------------------------------------------------------------------------------------
% nouveau point
\item \textbf{Lecture et adoption du procès-verbal de la réunion du 5 février 2013}

Le procès-verbal proposé est adopté à l'unanimité.


%--------------------------------------------------------------------------------------
% nouveau point
\item \textbf{Affaires découlant du procès-verbal}

\begin{enumerate}

\item Livrable 1 

Il faut faire attention à la qualité du français pour les prochains livrables.


\item Avancement du projet

\begin{itemize}

\item Les étudiants de génie électrique prévoient finir la partie électrique (alimentation, détection de l'antenne, asservissement et conception du circuit du préhenseur) la semaine prochaine soit le 19 février 2013.

\item Diane et Pierre-luc vont contrôler l'asservissement avec le microcontrôleur pour le 19 février 2013.

\item Imane va programmer la détection d'obstacle avec la kinect.
\end{itemize}

\end{enumerate}


%--------------------------------------------------------------------------------------
% nouveau point



%--------------------------------------------------------------------------------------
% nouveau point
\item \textbf{Divers}

Aucun élément à discuter.
\\

%--------------------------------------------------------------------------------------
\item \textbf{Répartitions des tâches pour livrable 2}

\begin{enumerate}

\item Daniel: Diagramme de fonctionnalités, registre de risques, plan de test, description des propriétés fonctionnelles, écriture de la partie avancement de la conception et de la construction du système.
\item Émile: Plan de test, écriture de la partie avancement de la conception et de la construction du système.
\item Francis:Diagramme physique, plan de test, écriture de la partie avancement de la conception et de la construction du système.
\item Diane: plan de test, tests unitaires, écriture de la partie avancement de la conception et de la construction du système.
\item Pierre-Luc: plan de test, tests unitaires, écriture de la partie avancement de la conception et de la construction du système.
\item Imane:Diagramme de classe (itération 2), plan de test, tests unitaires, écriture de la partie avancement de la conception et de la construction du système.
\item Philipe:Diagramme de séquences (itération 2), plan de test, tests unitaires, écriture de la partie avancement de la conception et de la construction du système.
\item Olivier:Plan de test, tests unitaires, écriture de la partie avancement de la conception et de la construction du système.
\item Tout le monde doit fournir la partie qu'il a préparé pour la présentation orale à Francis avant le 22 février.
\end{enumerate}


%--------------------------------------------------------------------------------------
% nouveau point
\item \textbf{Évaluation de la réunion}

Garder comme objectif des réunions concises qui ne s'écartent pas de l'objectif de départ.

%--------------------------------------------------------------------------------------
% nouveau point
\item \textbf{Date, heure, lieu et objectif de la prochaine réunion}

\begin{tabular}{@{}lll}
   Date: 2013/02/19
   & Heure: 08h30
   &  Lieu: Bibliothèque VCH
\end{tabular}
\par
Description de l'objectif ou des objectifs.

%--------------------------------------------------------------------------------------
% nouveau point
\item \textbf{Fermeture de la réunion}

Heure: 10h45


%--------------------------------------------------------------------------------------
% nouveau point
\item \textbf{Étaient présents}

\begin{dinglist}{"33}
   \item Daniel Thibodeau
   \item Pierre-Luc Buhler
   \item Francis Valois
   \item Diane Fournier
   \item Imane Mouhtij
   \item Émile Arsenault
\end{dinglist}

\end{enumerate}

\end{document}
%--------------------------------------------------------------------------------------
%---------------------------------- fin du document -----------------------------------
%--------------------------------------------------------------------------------------
