% !TEX encoding = UTF-8 Unicode
%%%%%%%%%%%%%%%%%%%%%%%%%%%%%%%%%%%%%%%%%%%%%%%%%%%%%%%%%%%%%%%%%%%%%%%%%%%%%%%%%%%%%%%
%%%%%%%%%%%%%%%%%%%%%%%%%%%%%%%%%%%%%%%%%%%%%%%%%%%%%%%%%%%%%%%%%%%%%%%%%%%%%%%%%%%%%%%
%%%%%%%%%%%%%                                                             %%%%%%%%%%%%%
%%%%%%%%%%%%%   Exemple d'ordre du jour                                   %%%%%%%%%%%%%
%%%%%%%%%%%%%   par Pierre Tremblay, Université Laval                     %%%%%%%%%%%%%
%%%%%%%%%%%%%   2007/12/07 - version 1.3                                  %%%%%%%%%%%%%
%%%%%%%%%%%%%                                                             %%%%%%%%%%%%%
%%%%%%%%%%%%%%%%%%%%%%%%%%%%%%%%%%%%%%%%%%%%%%%%%%%%%%%%%%%%%%%%%%%%%%%%%%%%%%%%%%%%%%%
%%%%%%%%%%%%%%%%%%%%%%%%%%%%%%%%%%%%%%%%%%%%%%%%%%%%%%%%%%%%%%%%%%%%%%%%%%%%%%%%%%%%%%%
%--------------------------------------------------------------------------------------
%------------------------------------- preambule --------------------------------------
%--------------------------------------------------------------------------------------
\documentclass[12pt]{ULojpv}
%**************************************************************************************
% Chargement des packages supplementaires
%
\usepackage[utf8]{inputenc}
%**************************************************************************************
%**************************************************************************************
% Definitions des parametres de l'en-tete
%
\Cours{GEL--3014 Design III }            % Nom du cours
\NumeroEquipe{05}                                     % Numero de l'equipe
\NomEquipe{Place holder}                               % Nom de l'equipe
\Objet{Ordre du jour}                                 % Nom du document
\SujetRencontre{Livrable 1 et mise à jour des tâches}             % Sujet de la rencontre
\DateRencontre{2013/02/12}                            % Date de la rencontre
\LocalRencontre{Bibliothèque}                            % Local de la rencontre
\HeureRencontre{08h30}                                % Heure de la rencontre
%**************************************************************************************
%--------------------------------------------------------------------------------------
%--------------------------------- corps du document ----------------------------------
%--------------------------------------------------------------------------------------
\begin{document}
\entete
\begin{enumerate}
   \item \textbf{Ouverture de la réunion}
   \item \textbf{Nomination du secrétaire}
   \item \textbf{Lecture et adoption de l'ordre du jour}(2 min)
   \item \textbf{Retour sur tâches effectuées pour le livrable 1}
   \item \textbf{Remplir formulaire d'évaluation individuelle}
   \item \textbf{Daniel: Flash meeting 1 (15-20 min)} 
   \item \textbf{Daniel: Revue côté électrique}
   \begin{itemize}
   \item Réalisation d'une alimentation 5V fonctionnelle 
   \item Achat de pièce pour alimentation plus robuste (50kHz)
   \item Planification implantation pratique
   \end{itemize}
   \item \textbf{Philippe: Revue côté logiciel}
   \item \textbf{Diane: Revue côté informatique}
   \item \textbf{Daniel: tâches pour l'équipe électrique}
   \begin{itemize}
   \item Le système d'acquisition de données des servomoteurs doit être fonctionnel le plus tôt possible Pierre-Luc Buhler (100\%)
   \item Daniel Thibodeau sera responsable de l'identification des moteurs (25\%), de la gestion (25\%) ainsi que la réalisation de l'alimentation (50\%)
   \item Francis Valois sera responsable de l'identification des moteurs (50\%) ainsi que de la réalisation de l'alimentation (50\%)
   \item Émile Arsenault sera responsable du circuit d'antenne (75\%) et de préhension (25\%)
   \end{itemize}
   \item \textbf{Philippe: tâches pour l'équipe logicielle}
   \item \textbf{Diane: tâches pour l'équipe informatique}
      \item \textbf{Prévision des tâches  pour la présentation orale du 28 février}
   \begin{itemize}
   \item Remise \textbf{ABSOLUE} des fichiers .pptx dans le format prédéfini par l'équipe (prêt vendredi le 15 février)
   \item Français et rigueur de mise.
   \end{itemize}
   \item \textbf{Prévision des tâches pour la remise du livrable 2 le 7 mars à 17h}
   \item \textbf{Évaluation de la réunion}
   \item \textbf{Date, heure, lieu et objectif de la prochaine réunion}
   \item \textbf{Fermeture du Flash Meeting}
\end{enumerate}

\end{document}
%--------------------------------------------------------------------------------------
%---------------------------------- fin du document -----------------------------------
%--------------------------------------------------------------------------------------
