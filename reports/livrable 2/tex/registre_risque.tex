%!TEX root = ../rapport.tex
%!TEX encoding = UTF-8 Unicode

% Chapitres "Introduction"

% modifié par Francis Valois, Université Laval
% 31/01/2011 - version 1.0 - Création du document
\section{Registre de risques}
\begin{landscape}
\begin{table}[htbp]
	\small
 	 \centering
  	\caption{Registre de risques partie 1}
  	\scalebox{0.8}{
	\tabcolsep=0.11cm
    \begin{tabular}{|c||p{3cm}|>{\centering\arraybackslash}m{1.5cm}|p{4cm}|p{4cm}|>{\centering\arraybackslash}m{1.5cm}|>{\centering\arraybackslash}m{1.5cm}|p{4cm}|p{3cm}|}\hline
    \textbf{Risque} &\textbf{Type de risque} &\textbf{Niveau de priorité (1- faible, 5-élevé)} & \textbf{Conséquences de l'occurrence du risque} & \textbf{Coût en performance associé au risque} & \textbf{Prob. d'occurrence (\%)} & \textbf{Coût estimé du risque (\$)}& \textbf{Plan de réduction du risque} & \textbf{Responsable du risque} \\\hline\hline
    \multirow{6}{*}{1} & Bris  de la batterie Li-Po suite à une mauvaise utilisation & 5 & Plus d'autonomie énergétique pour le robot, opération impossible & L'ensemble du système sera inopérable &15 & 60 & Formation des utilisateurs à l'endroit de l'utilisation. Apposition d'un capteur de tension. Dispositifs de protection (fusibles) & É. Arsenault \\\hline
    2 & Bris  d'un ou des système d'alimentation & 5 & Les systèmes auxiliaires, le Mac mini ou les moteurs ne seront pas alimentés. Le système ne sera pas fonctionnel & Une partie ou l'ensemble du système sera inopérable & 20 & 25 & Utilisateur de connecteurs protégés (d'un seul sens possible). Dispositifs de protection (fusibles), surdimensionnement des composantes d'alimentation & D. Thibodeau \\\hline
    3 & Bris  du crayon lors du dessin & 4 & Le dessin ne pourra être complété & La portion dessin sera partiellement complète & 20 & 2 & Tests rigoureux et optimisation du choix de crayon avant la compétition & É. Arsenault \\\hline
    4 & Bris du système de préhension & 5 & Le dessin ne pourra être complété et selon le moment du bris, la trajectoire du robot peut être affectée & La portion dessin sera partiellement complète et la trajectoire sera déviée & 15 & 10 & Tests rigoureux et essais multiples pour vérifier la stabilité en température lors du fonctionnement  & É. Arsenault \\\hline
    5 & Problèmes d'alimentation électrique (surtension ou baisse de tension, surchauffe ou ondulation importante) &4 & Une surtension importante pourrait être destructrice pour le Mac mini ou le microcontrôleur, une surchauffe causerait un bris des systèmes d'alimentation et une ondulation importante pourrait rendre le système instable & Conséquences des risques 1 et 2 en plus de bris du Mac mini ou du microcontrôleur & 10 & 100 & Tests rigoureux et essais multiples pour vérifier la stabilité en température, en tesion et en courant lors du fonctionnement  & D. Thibodeau \\\hline
    \end{tabular}}%
  \label{tab:rr1}%
\end{table}%
\end{landscape}
\begin{landscape}
\begin{table}[htbp]
	\small
 	 \centering
  	\caption{Registre de risques partie 2}
  	\scalebox{0.8}{
	\tabcolsep=0.11cm
    \begin{tabular}{|c||p{3cm}|>{\centering\arraybackslash}m{1.5cm}|p{4cm}|p{4cm}|>{\centering\arraybackslash}m{1.5cm}|>{\centering\arraybackslash}m{1.5cm}|p{4cm}|p{3cm}|}\hline
    \textbf{Risque} &\textbf{Type de risque} &\textbf{Niveau de priorité (1- faible, 5-élevé)} & \textbf{Conséquences de l'occurrence du risque} & \textbf{Coût en performance associé au risque} & \textbf{Prob. d'occurrence (\%)} & \textbf{Coût estimé du risque (\$)}& \textbf{Plan de réduction du risque} & \textbf{Responsable du risque} \\\hline\hline
    6 & Problèmes de réception du signal d'antenne & 4 & Si la réception est erronée, le robot peut exécuter sa séquence, mais l'exécution ne sera pas conforme au message. Si la réception est impossible, le système ne s'amorcera pas. & Accomplissement de la mauvaise tâche ou système non fonctionnel & 5 & 15 & Tests répétés pour un taux de succès de 100\% lors de la réception et du décodage. & D.Fournier \\\hline
    7 & Problème d'identification du robot (vision) & 5 & Correction de la trajectoire erronée, risque de rencontrer les obstacles & La trajectoire réelle ne sera pas optimale & 10 &  & Tests rigoureux et taux de succès de l'identification proche de 100\% & I. Mouhtij \\\hline
    8 & Problème de détection des chiffres du sudo cube & 5 & Erreur dans la résolution du sudo cube (mauvais chiffres à la base) & Le chiffre dessiné ne sera pas le bon & 5 &       & Tests répétés pour un taux de succès de près de 100\% lors de la détection des chiffres & P. Bourdages \\\hline
    9 & Problème de résolution du sudo cube & 5 & Erreur dans la résolution du sudo cube, emballement de l'algorithme et arrêt de la tâche & Le chiffre dessiné ne sera pas le bon, le robot peut arrêter sa séquence d'opération avant le moment prévu & 15 &       & Tests répétés pour un taux de succès de près de 100\% lors de la résolution du cube & O. Sylvain \\\hline
    10 & Problème de communication entre le Mac mini et la station de base & 5 & Les informations requises ne pourront être transmises correctement, on perd l'information sur le comportement du robot ainsi que sa position. Le robot ne pourra pas se localiser initialement et en temps réel. & Le robot de remplira pas les exigences d'affichage sur la station de base, le robot ne recevra aucune position de la Kinect  & 5 &       & Tests répétés pour un taux de succès de près de 100\% lors de la transmission et de la réception de l'information en temps réel entre le Mac mini et la station de base & P. Bourdages \\\hline
\end{tabular}}%
  \label{tab:rr2}%
\end{table}%
\end{landscape}
\begin{landscape}
\begin{table}[htbp]
	\small
 	 \centering
  	\caption{Registre de risques partie 3}
  	\scalebox{0.8}{
	\tabcolsep=0.11cm
    \begin{tabular}{|c||p{3cm}|>{\centering\arraybackslash}m{1.5cm}|p{4cm}|p{4cm}|>{\centering\arraybackslash}m{1.5cm}|>{\centering\arraybackslash}m{1.5cm}|p{4cm}|p{3cm}|}\hline
    \textbf{Risque} &\textbf{Type de risque} &\textbf{Niveau de priorité (1- faible, 5-élevé)} & \textbf{Conséquences de l'occurrence du risque} & \textbf{Coût en performance associé au risque} & \textbf{Prob. d'occurrence (\%)} & \textbf{Coût estimé du risque (\$)}& \textbf{Plan de réduction du risque} & \textbf{Responsable du risque} \\\hline\hline
	11 & Problème de communication entre le Mac mini et le microcontrôleur & 5 & L'ensemble des auxiliaires ne fonctionnera pas correctement. Le système sera non fonctionnel & Robot incapable de se déplacer selon la trajectoire prévue et d'effectuer la tâche & 5 &       & Tests répétés pour un taux de succès de près de 100\% lors de la transmission et de la réception de l'information entre le Mac mini et le microcontrôleur & D. Fournier \\\hline
    12 & Asservissement des moteurs déficient & 3 & Un asservissement instable causera des dépassements de positions et un temps de réponse plus long du moteur aux consignes. Dans le pire des cas, le robot deviendrait incontrôlable. & Robot plus lent ou incapable de se déplacer, hausse du risque de toucher les obstacles, système non fonctionnel & 10 &       &Tests rigoureux dans le plus de conditions possible. Maximiser la robustesse en réduisant le temps de réponse. & P. Buhler \\\hline
    13 & Contact avec un ou des obstacles & 3 & Bris du système et déviation de trajectoire possible & Un robot qui entre en contact avec les obstacles ne remplit pas les exigences du projet & 20 &       & Tests rigoureux sur les déplacements et marge de sécurité importante pour le contournement. & O. Sylvain \\\hline
    14 & Bris d'une portion ou de la totalité du microcontrôleur & 4 & Le bris d'une portion du microcontrôleur empêche l'exécution de la tâche dans son ensemble et peut occasionner des bris dans les systèmes reliés & Un robot qui ne peut pas se déplacer correctement, qui n'allume pas la DEL ou qui n'active pas le préhenseur & 5 & 100 & Isolation des entrées et sorties avec des dispositifs de protection (diode), limiteur de courant & D. Thibodeau \\\hline
    15 & Bris du pont en H & 5 & Les moteurs ne pourront être commandés correctement & Le robot ne peut pas se déplacer, le système n'est pas fonctionnel & 5 & 130 & Utilisation d'un régulateur de type PID afin d'éviter les appels brusques de courants, positionnement du pont de manière à limiter son exposition aux accrochages & F. Valois \\\hline
    \end{tabular}}%
  \label{tab:rr3}%
\end{table}%
\end{landscape}

