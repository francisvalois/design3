%!TEX root = ../rapport.tex
%!TEX encoding = UTF-8 Unicode

% Chapitres "Introduction"

% modifié par Francis Valois, Université Laval
% 31/01/2011 - version 1.0 - Création du document

\chapter{Plan de tests avec matrice de vérification des exigences intégrées}
\label{s:plantest}
\begin{landscape}
\begin{table}[htbp]
	\small
 	 \centering
  	\caption{Plan de tests côté matériel}
  	\scalebox{0.8}{
	\tabcolsep=0.11cm
    \begin{tabular}{|Z{\centering}{m}{4cm}||Z{\centering}{m}{5cm}|Z{\centering}{m}{5cm}|Z{\centering}{m}{3cm}|Z{\centering}{m}{4cm}|Z{\centering}{m}{5cm}|}\hline
    \textbf{Fonctionnalité}& \textbf{Sous-fonctionnalité} & \textbf{Exigence}& \textbf{Méthode de vérification}& \textbf{Équipement requis}& \textbf{Méthode d'analyse}\\ \hline\hline
    \multirow{4}{4cm}{\centering Traitement numérique} & \multirow{4}{5cm}{\centering Contrôler le robot pour le dessin}& \multirow{4}{5cm}{\centering Précision de $\pm$1cm de la ligne centrale}&\multirow{4}{3cm}{\centering Tests pratiques}& \multirow{1}{4cm}{\centering Robot fonctionnel}& \multirow{4}{5cm}{\centering Effectuer l'ensemble des dessins 3 fois et mesurer l'écart maximal}\\\cline{5-5}
    & & &  																											& Ruban à mesurer  & \\\cline{5-5}
    & & &  																											& Gabarits des dessins& \\\cline{5-5}
    & & &  																											& Crayon              &\\\hline
    \multirow{4}{4cm}{\centering Déplacement} & \multirow{4}{5cm}{\centering Se déplacer sans toucher aux obstacles}& \multirow{4}{5cm}{\centering Distance minimale de 1cm par rapport à l'axe de la trajectoire et vitesse supérieure à 3cm/s}&\multirow{4}{3cm}{\centering Tests pratiques}& \multirow{1}{4cm}{\centering Robot fonctionnel}& \multirow{4}{5cm}{\centering Vérifier la déviation maximale ainsi que la vitesse moyenne pour une dizaine de trajectoires différentes}\\\cline{5-5}
    & & &  																											& Ruban à mesurer  & \\\cline{5-5}
    & & &  																											& Crayon& \\\cline{5-5}
    & & & 																											& Rapporteur d'angle&\\\hline
   \end{tabular}}%
  \label{tab:rr5}%
\end{table}%
\end{landscape}