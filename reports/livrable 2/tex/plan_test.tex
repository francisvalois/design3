%!TEX root = ../rapport.tex
%!TEX encoding = UTF-8 Unicode

% Chapitres "Introduction"

% modifié par Francis Valois, Université Laval
% 31/01/2011 - version 1.0 - Création du document

\chapter{Plan de tests}
\label{s:plantest}
\begin{landscape}
\begin{table}[htbp]
	\small
 	 \centering
  	\caption{Plan de tests côté matériel (partie 1)}
  	\scalebox{0.8}{
	\tabcolsep=0.11cm
    \begin{tabular}{|Z{\centering}{m}{4cm}||Z{\centering}{m}{5cm}|Z{\centering}{m}{5cm}|Z{\centering}{m}{3cm}|Z{\centering}{m}{4cm}|Z{\centering}{m}{5cm}|}\hline
    \textbf{Fonctionnalité}& \textbf{Sous-fonctionnalité} & \textbf{Exigence}& \textbf{Méthode de vérification}& \textbf{Équipement requis}& \textbf{Méthode d'analyse}\\ \hline\hline
    
    \multirow{4}{4cm}{\centering Traitement numérique} & \multirow{4}{5cm}{\centering Contrôler le robot pour le dessin}& \multirow{4}{5cm}{\centering Précision de $\pm$1cm de la ligne centrale et temps d'exécution inférieur à 60s}&\multirow{4}{3cm}{\centering Tests pratiques}& \multirow{1}{4cm}{\centering Robot fonctionnel}& \multirow{4}{5cm}{\centering Effectuer l'ensemble des dessins 3 fois et mesurer l'écart maximal}\\\cline{5-5}
    & & &  																											& Ruban à mesurer  & \\\cline{5-5}
    & & &  																											& Gabarits des dessins& \\\cline{5-5}
    & & &  																											& Crayon              &\\\hline
    \multirow{4}{4cm}{\centering Déplacement} & \multirow{4}{5cm}{\centering Se déplacer sans toucher aux obstacles}& \multirow{4}{5cm}{\centering Distance minimale de 1cm par rapport à l'axe de la trajectoire et vitesse supérieure à 3cm/s}&\multirow{4}{3cm}{\centering Tests pratiques}& \multirow{1}{4cm}{\centering Robot fonctionnel}& \multirow{4}{5cm}{\centering Vérifier la déviation maximale ainsi que la vitesse moyenne pour une dizaine de trajectoires différentes}\\\cline{5-5}
    & & &  																											& Ruban à mesurer  & \\\cline{5-5}
    & & &  																											& Crayon& \\\cline{5-5}
    & & & 																											& Rapporteur d'angle&\\\hline
    
     \multirow{5}{4cm}{\centering Alimentation} & \multirow{5}{5cm}{\centering Alimenter le mac-mini et le préhenseur à une tension de 24 volts}& \multirow{5}{5cm}{\centering Fournir une tension de 24 volts CC avec des oscillations maximum de 200 mV}&\multirow{5}{3cm}{\centering Tests de mesure}& \multirow{1}{4cm}{\centering Mac mini}& \multirow{5}{5cm}{\centering Vérifier la tension moyenne à la sortie de l'alimentation et l'amplitude des oscillations qui y sont superposées sur l'oscilloscope}\\\cline{5-5}
    & & &  																											& \multirow{2}{4cm}{\centering Oscilloscope} & \\
    & & &  																											& & \\\cline{5-5}	
    & & &  																											& \multirow{2}{4cm}{\centering Sondes de mesure} & \\
    & & &  																											& & \\\hline											
    \multirow{5}{4cm}{\centering Alimentation} & \multirow{5}{5cm}{\centering Alimenter le micro-contrôleur et les différents circuits à une tension de 5 volts}& \multirow{5}{5cm}{\centering Fournir une tension de 5 volts CC avec des oscillations maximum de 0.2 volt}&\multirow{5}{3cm}{\centering Tests de mesure}& \multirow{1}{4cm}{\centering Robot fonctionnel}& \multirow{5}{5cm}{\centering Vérifier la tension moyenne à la sortie de l'alimentation et l'amplitude des oscillations qui y sont superposées sur l'oscilloscope}\\\cline{5-5}
    & & &  																											& \multirow{2}{4cm}{\centering Oscilloscope} & \\
    & & &  																											& & \\\cline{5-5}	
    & & &  																											& \multirow{2}{4cm}{\centering Sondes de mesure} & \\
    & & &  																											& & \\\hline     
     
     \multirow{4}{4cm}{\centering Récepteur manchester} & \multirow{4}{5cm}{\centering Recevoir le signal manchester et le rendre utilisable par le micro-contrôleur}& \multirow{4}{5cm}{\centering Fournir un signal carré 0-5 volts à l'image du signal capté}&\multirow{4}{3cm}{\centering Tests de mesure}& \multirow{1}{4cm}{\centering Robot fonctionnel}& \multirow{3}{5cm}{\centering Vérifier la présence du signal carré en sortie du module de réception lorsque le robot est à proximité de l'antenne }\\\cline{5-5}
    & & &  																											& Oscilloscope & \\\cline{5-5}
     & & &  																											& Antenne fonctionnelle & \\\cline{5-5}
    & & &  																											& Sondes de mesure & \\\hline
     \multirow{5}{4cm}{\centering Préhenseur} & \multirow{5}{5cm}{\centering Monter et descendre le crayon pour passer en mode écriture ou arrêt d'écriture}& \multirow{5}{5cm}{\centering Descendre le crayon lorsqu'une commande 5V est apliqué et le remonté lorsque la commande 0V est appliquée}&\multirow{5}{3cm}{\centering Test pratique}& \multirow{1}{4cm}{\centering Robot fonctionnel}& \multirow{5}{5cm}{\centering Vérifier que le crayon descend lorsqu'on applique la commande de 5V au circuit et qu'il remonte lorsqu'on la relâche}\\\cline{5-5}
    & & &  																											& \multirow{4}{4cm}{\centering Source 24V CC} & \\\
    & & &  																											&  & \\\
    & & & & & 									
     \\\
    & & & & & 																										\\\hline
    \multirow{5}{4cm}{\centering Affichage} & \multirow{5}{5cm}{\centering Allumer la DEL lorsque la tâche est complétée}& \multirow{5}{5cm}{\centering Allumer la DEL lorsque 5V est appliqué à son circuit et l'éteindre lorsqu'on relâche la commande}&\multirow{5}{3cm}{\centering Test électrique}& \multirow{1}{4cm}{\centering Robot fonctionnel}& \multirow{5}{5cm}{\centering Vérifier que la DEL verte allume lorsqu'on applique une commande de 5V à son circuit}\\\cline{5-5}
    & & &  																											& \multirow{4}{4cm}{\centering Source 5V CC} & \\\
    & & &  																											&  & \\\
    & & & & & 									
     \\\
    & & & & & 																										\\\hline
\end{tabular}}%
  \label{tab:pt1}%
\end{table}%
\end{landscape}
    
    \begin{landscape}
\begin{table}[htbp]
	\small
 	 \centering
  	\caption{Plan de tests côté matériel (partie 2)}
  	\scalebox{0.8}{
	\tabcolsep=0.11cm
    \begin{tabular}{|Z{\centering}{m}{4cm}||Z{\centering}{m}{5cm}|Z{\centering}{m}{5cm}|Z{\centering}{m}{3cm}|Z{\centering}{m}{4cm}|Z{\centering}{m}{5cm}|}\hline
    \textbf{Fonctionnalité}& \textbf{Sous-fonctionnalité} & \textbf{Exigence}& \textbf{Méthode de vérification}& \textbf{Équipement requis}& \textbf{Méthode d'analyse}\\ \hline\hline
     																								
    
    \multirow{4}{4cm}{\centering Alimentation} & \multirow{4}{5cm}{\centering Utiliser une pile rechargeable}& \multirow{4}{5cm}{\centering Avoir un temps de charge (pour 10 minutes) inférieur à 1 heure}& \multirow{4}{3cm}{\centering Test de mesure}& \multirow{1}{4cm}{\centering Pile}& \multirow{4}{5cm}{\centering Mesurer le temps de charge pour une opération de 10 minutes}\\\cline{5-5}
    & & & & Testeur de batterie & \\\cline{5-5}
    & & & & Chargeur & \\\cline{5-5}
    & & & & Chronomètre & \\\hline
\multirow{6}{4cm}{\centering Communication Microcontrôleur - Mac mini} & \multirow{6}{5cm}{\centering Transmettre les commandes du Mac mini vers le microcontôleur via un port série}& \multirow{6}{5cm}{\centering Vitesse de transfert d'environ 57600 bit/s}&\multirow{6}{3cm}{\centering Test pratique}& \multirow{6}{4cm}{\centering Robot fonctionnel pour ses déplacements}& \multirow{6}{5cm}{\centering Vérifier que les commandes passent du Mac mini au microcontrôleur et que les mécanismes de vérification et de gestion des erreurs de communication sont efficaces}\\
    & & & & & \\\
    & & & & & \\\
    & & & & & \\\
    & & & & & \\\
    & & & & & \\\hline     
   \multirow{7}{4cm}{\centering Affichage des informations sur le robot} & \multirow{7}{5cm}{\centering Afficher les paramètres décodé de l'antenne par le microcontrôleur sur un écran situé sur le robot}& \multirow{7}{5cm}{\centering Afficher les paramètres dès qu'ils sont décodés à partir du signal d'antenne.}& \multirow{7}{3cm}{\centering Test pratique}& \multirow{1}{4cm}{\centering Robot fonctionnel}& \multirow{7}{5cm}{\centering Vérifier que les informations affichées sont lisibles et qu'elles concordent avec les informations décodées qui se trouvent dans la mémoire du microcontrôleur.}\\\cline{5-5}
    & & & & \multirow{6}{4cm}{\centering Câble USB} & \\\
    & & & & & \\\
    & & & & & \\\
    & & & & & \\\
    & & & & & \\\
    & & & & & \\\hline
    
    \multirow{5}{4cm}{\centering Positionnement de la caméra} & \multirow{5}{5cm}{\centering Garder la camera en position malgré les perturbations extérieures}& \multirow{5}{5cm}{\centering Replacer la camera à sa position de départ lorsqu'on allume l'alimentation 5V}&\multirow{5}{3cm}{\centering Test de manipulation}& \multirow{5}{4cm}{\centering Robot fonctionnel}& \multirow{5}{5cm}{\centering Déplacer la camera lorsque l'alimentation est fermé et vérifier qu'elle reprend sa position lorsqu'on actionne l'alimentation 5V}\\
    & & & & &\\  																											
    & & & & &\\ 																											
    & & & & &\\ 
    & & & & &\\\hline 											



   \end{tabular}}%
  \label{tab:pt2}%
\end{table}%
\end{landscape}

    \begin{landscape}
\begin{table}[htbp]
  \small
   \centering
    \caption{Plan de tests côté logiciel}
    \scalebox{0.8}{
  \tabcolsep=0.11cm
    \begin{tabular}{|Z{\centering}{m}{4cm}||Z{\centering}{m}{5cm}|Z{\centering}{m}{5cm}|Z{\centering}{m}{3cm}|Z{\centering}{m}{4cm}|Z{\centering}{m}{5cm}|}\hline
    \textbf{Fonctionnalité}& \textbf{Sous-fonctionnalité} & \textbf{Exigence}& \textbf{Méthode de vérification}& \textbf{Équipement requis}& \textbf{Méthode d'analyse}\\ \hline\hline
     \multirow{15}{4cm}{\centering Vision numérique}   &  \multirow{7}{5cm}{ \centering Détecter les obstacles}  & \multirow{7}{5cm}{ \centering La position des obstacles obtenue à l'aide de la Kinect doit être précise au centimètre près.Le temps de calcul doit être inférieur à 1s}  & \multirow{7}{3cm}{ \centering Test de mesure}   & \multirow{2}{4cm}{\centering Station de base}     & \multirow{4}{5cm}{\centering Mesurer la distance réelle en X et Y pour différentes position d'obstacles avec et sans obstruction et la comparer avec la position vu par le logiciel. Mesurer le temps d'exécution du logiciel} \\ 
                                                      &                                                         &                                                                                                                               &                                                 &                                                   & \\ \cline{5-5} 
                                                      &                                                         &                                                                                                                               &                                                 & Ruban à Mesurer                                   & \\ \cline{5-5} 
                                                      &                                                         &                                                                                                                               &                                                 & Kinect                                            & \\ \cline{5-5} 
                                                      &                                                         &                                                                                                                               &                                                 & Algorithme de détection des obstacle              & \\ \cline{5-5}
                                                      &                                                         &                                                                                                                               &                                                 & Table avec obstacles                              & \\ \cline{2-6}                                                                                                     
                                                      &  \multirow{8}{5cm}{ \centering Détecter le robot}       & \multirow{7}{5cm}{ \centering La position du robot obtenue à l'aide de la Kinect doit être précise au centimètre près. Le temps de calcul doit être inférieur à 1s}        & \multirow{7}{3cm}{ \centering Test de mesure}   & \multirow{2}{4cm}{\centering Station de base}     & \multirow{4}{5cm}{\centering Mesurer la distance réelle en X et Y du robot pour différentes position d'obstacles avec et sans obstruction et la comparer avec la position du robot vu par le logiciel. Mesurer le temps d'exécution du logiciel} \\ 
                                                      &                                                         &                                                                                                                               &                                                 &                                                   & \\ \cline{5-5} 
                                                      &                                                         &                                                                                                                               &                                                 & Ruban à Mesurer                                   & \\ \cline{5-5} 
                                                      &                                                         &                                                                                                                               &                                                 & Kinect                                            & \\ \cline{5-5} 
                                                      &                                                         &                                                                                                                               &                                                 & Algorithme de détection du robot                  & \\ \cline{5-5}
                                                      &                                                         &                                                                                                                               &                                                 & Table avec obstacles                              & \\ \cline{5-5}
                                                      &                                                         &                                                                                                                               &                                                 & Robot                                             & \\ \cline{1-6}
    \multirow{4}{4cm}{\centering Traitement numérique} & \multirow{4}{5cm}{\centering Décoder le signal Manchester}& \multirow{4}{5cm}{\centering Décodage de toutes les combinaisons possibles en tolérant une variation de taux de transmission de $\pm$ 50 bits/s }&\multirow{4}{3cm}{\centering Tests pratiques}& \multirow{1}{4cm}{\centering Antenne}& \multirow{4}{5cm}{\centering Tester l'algorithme dans tous les cas possibles avec variation de taux de transmission de $\pm$ 50 bits/s}\\\cline{5-5}
 & & &  																											& \multirow{2}{4cm}{\centering Récepteur Manchester} & \\
    & & &  																											& & \\\cline{5-5}	
    & & &  																											& \multirow{2}{4cm}{\centering Microcontrôleur} & \\
    & & &  																											& & \\\hline     
    \multirow{5}{4cm}{\centering Traitement numérique} & \multirow{5}{5cm}{\centering Calculer la trajectoire optimale}& \multirow{5}{5cm}{\centering Calculer la trajectoire optimale en moins de 2s}&\multirow{5}{3cm}{\centering Tests Pratiques}& \multirow{5}{4cm}{\centering Ordinateur}& \multirow{5}{5cm}{\centering Vérifier l'efficacité et le temps de calcul de l'algorithme avec un jeu de positions du robot et des obstacles}\\
   & & & & & \\\
   & & & & & \\\
   & & & & & \\\
   & & & & & \\\hline   
    \multirow{4}{4cm}{\centering Traitement numérique} & \multirow{4}{5cm}{\centering Résoudre le sudocube}& \multirow{4}{5cm}{\centering Être capable de résoudre des sudocube de 7 chiffres et plus}&\multirow{4}{3cm}{\centering Tests pratiques}& \multirow{4}{4cm}{\centering Ordinateur}& \multirow{4}{5cm}{\centering Tester l'algorithme sur un jeu de sudocubes ayant des nombres différents de chiffres}\\
   & & & & & \\\
   & & & & & \\\
   & & & & & \\\hline  
       \multirow{5}{4cm}{\centering Vision numérique} & \multirow{5}{5cm}{\centering Lire le sudocube}& \multirow{5}{5cm}{\centering Le robot doit être situé entre (25.4 et 38.4)cm de distance par rapport au sudocube pour bien lire les informations}&\multirow{5}{3cm}{\centering Tests de mesure}& \multirow{1}{4cm}{\centering Algorithme de lecture}& \multirow{5}{5cm}{\centering Mesurer la distance entre le sudocube et la caméra après un déplacement vers un sudocube à analyser.}\\\cline{5-5}
   & & & & Robot fonctionnel & \\\cline{5-5}
    & & & & Caméra Web & \\\cline{5-5}
    & & & & Ruban à mesurer & \\\cline{5-5}
    & & & & Table de jeu & \\\hline

      \multirow{4}{4cm}{\centering Orientation} & \multirow{4}{5cm}{\centering Déterminer l'angle et l'orientation du robot}& \multirow{4}{5cm}{\centering L'orientation du robot doit être précise au degré près}&\multirow{4}{3cm}{\centering Tests de mesure}& \multirow{1}{4cm}{\centering Algo. d'orientation}& \multirow{4}{5cm}{\centering Trouver l'angle avec l'algorithme et comparer ce résultat avec le résultat calculé manuellement.}\\\cline{5-5}
    & & & & Caméra embarquée & \\\cline{5-5}
    & & & & Station de base & \\\cline{5-5}
    & & & & Table et robot & \\\hline

   \end{tabular}}%
  \label{tab:pt2}%
\end{table}%
\end{landscape}

