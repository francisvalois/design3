%!TEX root = ../rapport.tex
%!TEX encoding = UTF-8 Unicode

% Chapitres "Introduction"

% modifié par Francis Valois, Université Laval
% 31/01/2011 - version 1.0 - Création du document
\chapter{Description des propriétés fonctionnelles}
\label{s:fonctionnelles}
Pour simplifier la lecture du tableau de la description des propriétés fonctionnelles, celui-ci a été séparé en trois pages selon trois différentes sections: vision et traitement numérique (présenté dans le tableau \ref{tab:dpf1}), communication et déplacement (présenté dans le tableau \ref{tab:dpf2}) ainsi qu'alimentation et affichage (présenté dans le tableau \ref{tab:dpf3}). 

\newlength{\hcolw}
\setlength{\hcolw}{\textwidth}
\eject \pdfpagewidth=15.7in \pdfpageheight=10in
\textwidth=13.7in

\begin{table}[!ht]
\centering
	\begin{minipage}[c]{13.8in}
	\caption{Description des propriétés fonctionnelles: section "Vision et Traitement Numérique"} 
	\label{tab:dpf1}
	\small
	\scalebox{0.8}{
	\tabcolsep=0.11cm
	\begin{tabular}{|Z{\raggedright}{m}{6.5cm}||Z{\centering}{m}{1.5cm}|Z{\centering}{m}{1.5cm}|Z{\centering}{m}{1.5cm}|Z{\centering}{m}{1.5cm}|Z{\centering}{m}{1.5cm}|Z{\centering}{m}{1.5cm}|Z{\centering}{m}{2cm}|Z{\centering}{m}{1.5cm}|Z{\centering}{m}{1.5cm}|Z{\centering}{m}{1.5cm}|Z{\centering}{m}{1.5cm}|Z{\centering}{m}{2.3cm}|Z{\centering}{m}{1.5cm}|Z{\centering}{m}{2cm}|Z{\centering}{m}{2cm}|Z{\centering}{m}{3cm}|Z{\centering}{m}{2cm}|Z{\centering}{m}{2cm}|} 
		
		\cline{2-19}
		\multicolumn{1}{c|}{} 																						& \multicolumn{18}{c|}{\textbf{Fonctionnalités}} \\ \cline{2-19}
		\multicolumn{1}{c|}{} 																						& \multicolumn{10}{c|}{Vision numérique} 																		& \multicolumn{8}{c|}{Traitement numérique}																																											\\ \cline{2-19}
		\multicolumn{1}{c|}{} 																						& \multicolumn{3}{Z{\centering}{m}{5cm}|}{Détecter obstacles} 	& \multicolumn{4}{Z{\centering}{m}{7cm}|}{Localiser le robot} 		& \multicolumn{3}{Z{\centering}{m}{5cm}|}{Lire le cube} 			&  \multicolumn{2}{Z{\centering}{m}{4cm}|}{Calculer la trajectoire optimale}	& \multicolumn{2}{Z{\centering}{m}{4cm}|}{Contrôler le robot pour le dessin} 	& Décoder le signal d'antenne 	& Choisir le cube selon le signal d'antenne & \multicolumn{2}{Z{\centering}{m}{4cm}|}{Résoudre le sudocube}  \\ \hline
		\centering\textbf{Exigences du client}																		& Temps de calcul (s) & Précision en X (cm) & Précision en Z (cm) & Temps de calcul (s) & Précision en X (cm) & Précision en Z (cm) & Précision angulaire ($^o$)  & Temps de calcul (s) 	& Distance min (m) & Distance max(m) & Temps de calcul (s) & Nombre de déplacement maximum & Temps de calcul (s) 	& Précision (cm) 									& 	Nombre de périodes nécessaires	& 											& Temps de calcul (s) 	& Chiffres initiaux minimum\\ \hline  \hline
		Être autonome pendant un minimum de 10 minutes 																& 2 & 4 & 4 & 2 & 5 & 5 & 4 & 5 & 3 & 1 & 2 & 2 & 3 & 3 & 2 & 4 & 2 & 1 \\ \hline
		Se déplacer selon la trajectoire optimale																	& 3 & 5 & 5 & 3 & 5 & 5 & 5 &  &  &  & 5 & 5 &  &  & 3 & 3 &  &  \\ \hline
		Effectuer une séquence complète en moins de 10 minutes 														& 5 & 3 & 3 & 5 & 3 & 3 & 3 & 5 & 3 & 1 & 5 & 3 & 5 &  &  & 3 & 3 & 2 \\ \hline
		Alimenter le Mac mini avec une tension de 22V à 30V et ondulation de tension inférieure à 200 mV 			&  &  &  &  &  &  &  &  &  &  &  &  &  &  &  &  &  &  \\ \hline
		Résoudre sudocube 																							&  &  &  &  &  &  &  & 5 & 3 & 1 &  &  &  &  &  &  & 5 & 5 \\ \hline
		Dessiner le chiffre selon le signal d'antenne dans une zone prédéfinie (jaune) avec une précision de ± 1cm 	&  &  &  &  &  &  &  &  &  &  &  &  &  & 5 &  &  &  &  \\ \hline
		Éviter les obstacles ainsi que les murs de la table															& 3 & 5 & 5 & 3 & 5 & 5 & 5 &  &  &  & 3 & 5 &  &  &  &  &  &  \\ \hline
		Décoder le signal d'antenne																					&  &  &  &  &  &  &  &  &  &  &  &  &  &  & 5 &  &  &  \\ \hline
		Analyser le bon cube selon le signal d'antenne 																&  &  &  &  &  &  &  &  &  &  &  &  &  &  &  & 5 & 5 & 1 \\ \hline
		Utiliser la communication sans fil 																			&  &  &  &  &  &  &  &  &  &  &  &  &  &  &  &  &  &  \\ \hline
		Concevoir un système de préhension pour le crayon 															&  &  &  &  &  &  &  &  &  &  &  &  &  & 3 &  &  &  &  \\ \hline 
		Afficher position réelle																					&  &  &  & 3 & 5 & 5 & 5 &  &  &  &  &  &  &  &  &  &  &  \\ \hline
		Afficher la trajectoire optimale 																			& 3 & 5 & 5 & 3 & 5 & 5 & 5 &  &  &  & 3 & 5 &  &  & 3 & 5 &  &  \\ \hline
		Afficher le cube résolu et le chiffre de la case rouge sur la base 											&  &  &  &  &  &  &  & 5 & 2 & 2 &  &  &  &  &  &  & 5 & 5 \\ \hline
		Allumer une DEL lorsque tâche terminée 																		&  &  &  &  &  &  &  &  &  &  &  &  &  &  &  &  &  &  \\ \hline 
		Afficher message de fin 																					&  &  &  &  &  &  &  &  &  &  &  &  &  &  &  &  &  &  \\ \hline
		Afficher message de départ 																					&  &  &  &  &  &  &  &  &  &  &  &  &  &  &  &  &  &  \\ \hline
		Afficher trajectoire réelle avec un délai maximum de 15s 													&  &  &  & 3 & 5 & 5 & 5 &  &  &  &  &  &  &  &  &  &  &  \\ \hline
		Afficher les informations sur le robot 																		&  &  &  &  &  &  &  &  &  &  &  &  &  &  &  &  &  &  \\ \hline
		Le robot doit se retirer de la zone de dessin une fois celui-ci terminée									&  &  &  &  &  &  &  &  &  &  &  &  &  &  &  &  &  &  \\ \hline 
		Respecter un budget de 250\$ 																				&  &  &  &  &  &  &  &  &  &  &  &  & 3 & 5 & 3 &  &  &  \\ \hline 
		Respecter l'échéancier																						& 3 & 5 & 5 & 3 & 5 & 5 & 3 & 5 & 1 & 1 & 3 & 3 & 1 & 5 & 3 & 3 & 5 & 3 \\ \hline
		Avoir le temps d'exécution de la tâche																		&  &  &  &  &  &  &  &  &  &  &  &  &  &  &  &  &  &  \\ \hline
	\end{tabular}}
	\end{minipage}
\end{table}

\newpage
\eject \pdfpagewidth=15.7in 

\begin{table}[!ht]
\makebox[\textwidth][c]{
\begin{minipage}[c]{12.5in}
	\caption{Description des propriétés fonctionnelles: section "Communication et Déplacement"} 
	\label{tab:dpf2}
	\small
	\scalebox{0.8}{
	\tabcolsep=0.11cm
	\begin{tabular}{|Z{\raggedright}{m}{6.5cm}||Z{\centering}{m}{2cm}|Z{\centering}{m}{2cm}|Z{\centering}{m}{2cm}|Z{\centering}{m}{4cm}|Z{\centering}{m}{1.5cm}|Z{\centering}{m}{1.5cm}|Z{\centering}{m}{4cm}|Z{\centering}{m}{4cm}|Z{\centering}{m}{1.5cm}|Z{\centering}{m}{1.5cm}|Z{\centering}{m}{2.5cm}|Z{\centering}{m}{2cm}|Z{\centering}{m}{1.5cm}|} 
		
		\cline{2-14}
		\multicolumn{1}{c|}{} 																						& \multicolumn{13}{c|}{\textbf{Fonctionnalités}} \\ \cline{2-14}
		\multicolumn{1}{c|}{} 																						& \multicolumn{11}{c|}{Communication} & \multicolumn{2}{c|}{Déplacement}\\ \cline{2-14}
		\multicolumn{1}{c|}{} 																						& \multicolumn{1}{Z{\centering}{m}{2cm}|}{Recevoir le signal d'antenne} & \multicolumn{2}{Z{\centering}{m}{4cm}|}{Communiquer entre le robot et la station de base}  & Communiquer entre le Mac mini et le microcontrôleur & \multicolumn{2}{Z{\centering}{m}{3cm}|}{Commander les moteurs} & Transmettre les images de la Kinect vers le Mac & Transmettre les images de la caméra vers le Mac & \multicolumn{2}{Z{\centering}{m}{3cm}|}{Contrôler la position de la caméra}  & Commander le préhenseur du crayon & \multicolumn{2}{Z{\centering}{m}{3.5cm}|}{Se déplacer sans toucher aux obstacles} \\ \hline
		\centering\textbf{Exigences du client}																		&  & Vitesse (Mo/s) & Latence (ms) &Vitesse (bits/s) &  Précision (cm) &  Temps de réponse (s) & Taux de transfert (images/s) & Taux de transfert(images/s) & Temps de réponse (s) & Précision (degré) & Temps de réponse (s) & Résolution (Degrés) & Vitesse (m/s) \\ \hline  \hline
		Être autonome pendant un minimum de 10 minutes 																& 4     & 3     & 3     & 5     & 5     & 5     & 5     & 5     & 1     & 1     & 4     & 3     & 1 \\ \hline
    	Se déplacer selon la trajectoire optimale 																	& 3     &       &       & 3     & 5     & 5     & 5     & 2     &       &       &       & 5     & 5 \\ \hline
    	Effectuer une séquence complète en moins de 10 minutes 														& 5     &       &       & 3     & 4     & 3     & 5     &       &       &       &       & 5     & 5 \\ \hline
  		Alimenter le Mac mini avec une tension de 22V à 30V et ondulation de tension inférieure à 200 mV 			&       &       &       &       &       &       &       &       &       &       &       &       &  \\ \hline
    	Résoudre sudocube 																							&       &       &       &       &       &       &       & 3     & 1     & 1     &       &       &  \\ \hline
 		Dessiner le chiffre selon le signal d'antenne dans une zone prédéfinie (jaune) avec une précision de ± 1cm 	&       &       &       & 5     & 5     & 5     & 3     &       &       &       & 5     & 5     & 5 \\ \hline
    	Éviter les obstacles ainsi que les murs de la table 														&       &       &       & 5     & 3     & 3     & 5     & 4     &       &       &       & 5     &  \\ \hline
    	Décoder le signal d'antenne 																				&       &       &       & 2     &       &       &       &       &       &       &       &       &  \\ \hline
    	Analyser le bon cube selon le signal d'antenne 																& 5     &       &       &       &       &       &       &       &       &       &       &       &  \\ \hline
    	Utiliser la communication sans fil  																		&       & 5     & 5     &       &       &       & 5     &       &       &       &       &       &  \\ \hline
    	Concevoir un système de préhension pour le crayon  															&       &       &       &       &       &       &       &       &       &       & 5     &       &  \\ \hline
    	Afficher position réelle 																					&       & 5     & 5     & 5     &       &       & 5     &       &       &       &       &       &  \\ \hline
    	Afficher la trajectoire optimale 																			& 3     & 5     & 2     & 3     &       &       & 5     &       &       &       &       &       &  \\ \hline
    	Afficher le cube résolu et le chiffre de la case rouge sur la base 											&       & 5     & 2     &       &       &       &       & 3     &       &       &       &       &  \\ \hline
    	Allumer une DEL lorsque tâche terminée 																		&       &       &       & 5     &       &       &       &       &       &       &       &       &  \\ \hline
    	Afficher message de fin 																					&       & 5     & 1     &       &       &       &       &       &       &       &       &       &  \\ \hline
    	Afficher message de départ 																					&       & 5     & 1     &       &       &       &       &       &       &       &       &       &  \\ \hline
    	Afficher trajectoire réelle avec un délai maximum de 15s 													&       & 5     & 5     & 5     &       &       & 5     &       &       &       &       &       &  \\ \hline
    	Afficher les informations sur le robot 																		&       &       &       & 5     &       &       &       &       &       &       &       &       &  \\ \hline
    	Le robot doit se retirer de la zone de dessin une fois celui-ci terminé 									&       &       &       & 5     & 5     & 1     &       &       &       &       &       & 5     &  \\ \hline
    	Respecter un budget de 250\$ 																				& 3     &       &       & 1     & 5     & 1     &       &       &       &       & 2     &       &  \\ \hline 
    	Respecter l'échéancier 																						& 3     & 3     & 2     & 3     & 5     & 2     & 5     & 3     & 1     & 1     & 5     & 3     & 3 \\ \hline
		Avoir le temps d'exécution de la tâche																		&       &       &       &       &       &       &       &       &       &       &       &       &\\ \hline
	\end{tabular}}
	\end{minipage}}
\end{table}

\newpage
\eject \pdfpagewidth=15.7in 

\begin{table}[!ht]
\makebox[\textwidth][c]{
\begin{minipage}[c]{13.5in}
	\caption{Description des propriétés fonctionnelles: section "Alimentation et affichage"} 
	\label{tab:dpf3}
	\small
	\scalebox{0.8}{
	\tabcolsep=0.11cm
	\begin{tabular}{|Z{\raggedright}{m}{6.5cm}||Z{\centering}{m}{1.5cm}|Z{\centering}{m}{2cm}|Z{\centering}{m}{2.5cm}|Z{\centering}{m}{2cm}|Z{\centering}{m}{1.8cm}|Z{\centering}{m}{2.2cm}|Z{\centering}{m}{2cm}|Z{\centering}{m}{2cm}|Z{\centering}{m}{2cm}|Z{\centering}{m}{2cm}|Z{\centering}{m}{2.1cm}|Z{\centering}{m}{2cm}|Z{\centering}{m}{3cm}|Z{\centering}{m}{1.5cm}|Z{\centering}{m}{1.5cm}|Z{\centering}{m}{1.5cm}|} 
		
		\cline{2-17}
		\multicolumn{1}{c|}{} 																						& \multicolumn{16}{c|}{\textbf{Fonctionnalités}} \\ \cline{2-17}
		\multicolumn{1}{c|}{} 																						& \multicolumn{8}{c}{Alimentation} & \multicolumn{8}{|c|}{Affichage} \\ \cline{2-17}
		\multicolumn{1}{c|}{} 																						& \multicolumn{3}{Z{\centering}{m}{6.5cm}|}{Utiliser une pile rechargeable} & Alimenter les moteurs & \multicolumn{2}{Z{\centering}{m}{4.3cm}|}{Alimenter le Mac} & \multicolumn{2}{Z{\centering}{m}{4.5cm}|}{Alimenter les différents périphériques }& Afficher message d'initiation de la tâche & Afficher le cube résolu ainsi que la case rouse & Allumer la DEL lorsque tâche complétée & Afficher la trajectoire optimale & Afficher la position et trajectoire réelle & Afficher message de fin & Afficher sur le LCD & Afficher le temps d'exécution\\ \hline
		\centering\textbf{Exigences du client}																		& Énergie (Wh) & Courant maximal (A) & Durée d'une charge (min) & Puissance (W) & Puissance (W) & Ondulation de tension (mV) & Puissance (W) & Ondulation de tension (mV) & Puissance (W) &  &  &  & Temps d'actualisation (s) &  &  &  \\ \hline  \hline
		Être autonome pendant un minimum de 10 minutes 																& 5     & 3     & 5     & 3     & 3     & 3     & 3     & 5     &       &       &       &       &       &       &	& 5 \\ \hline
    	Se déplacer selon la trajectoire optimale 																	& 5     & 3     & 1     & 5     & 5     & 5     & 3     & 3     &       &       &       &       &       &       &  	&\\ \hline
    	Effectuer une séquence complète en moins de 10 minutes 														& 5     & 3     & 2     & 5     & 5     & 5     & 5     & 5     &       &       &       &       &       &       &  	&\\ \hline
    	Alimenter le Mac mini avec une tension de 22V à 30V et ondulation de tension inférieure à 200 mV			& 5     & 5     & 3     &       & 5     & 5     & 3     & 3     &       &       &       &       &       &       &  	&\\ \hline
    	Résoudre sudo-cube 																							& 5     & 5     & 2     &       & 5     & 5     & 5     & 5     &       &       &       &       &       &       &  	&\\ \hline
    	Dessiner le chiffre selon le signal d'antenne dans une zone prédéfinie (jaune) avec une précision de ± 1cm 	& 3     & 3     & 1     & 3     & 3     & 3     & 3     & 3     &       &       &       &       &       &       &  	&\\ \hline
    	Éviter les obstacles ainsi que les murs de la table 														& 3     & 3     & 1     & 3     & 3     & 3     & 3     & 3     &       &       &       &       &       &       &  	&\\ \hline
	    Décoder le signal d'antenne 																				& 5     & 3     & 1     &       & 3     & 3     & 5     & 5     &       &       &       &       &       &       & 3 &\\ \hline
	    Analyser le bon cube selon le signal d'antenne 																& 5     & 2     & 1     &       & 1     & 2     & 3     & 3     & 3     & 3     & 3     &       &       &       &  	&\\ \hline
	    Utiliser la communication sans fil  																		&       &       &       &       &       &       &       &       &       &       &       &       &       &       &  	&\\ \hline
	    Concevoir un système de préhension pour le crayon  															&       &       &       &       &       &       &       &       &       &       &       &       &       &       &  	&\\ \hline
	    Afficher position réelle 																					&       &       &       &       & 3     & 2     & 3     & 3     &       &       &       &       & 5     &       & 5 &\\ \hline
	    Afficher la trajectoire optimale 																			& 3     & 1     & 1     &       & 1     & 1     & 3     & 3     &       &       &       & 5     &       &       &  	&\\ \hline
	    Afficher le cube résolu et le chiffre de la case rouge sur la base 											& 1     & 1     & 1     &       & 1     & 1     &       &       &       & 3     &       &       &       &       &  	&\\ \hline
	    Allumer une DEL lorsque tâche terminée 																		& 3     & 1     & 1     &       & 3     & 2     & 5     & 5     &       &       & 5     &       &       &       &  	&\\ \hline
	    Afficher message de fin 																					& 3     & 1     & 1     &       & 3     & 2     &       &       &       &       &       &       &       & 5     &  	&\\ \hline
	    Afficher message de départ 																					& 3     & 1     & 1     &       & 3     & 2     &       &       & 5     &       &       &       &       &       &  	&\\ \hline
	    Afficher trajectoire réelle avec un délai maximum de 15s 													&       &       &       &       & 3     & 2     & 3     & 3     &       &       &       &       & 5     &       &  	&\\ \hline
	    Afficher les informations sur le robot 																		& 3     & 1     & 1     &       &       & 2     & 3     & 3     &       &       &       &       &       &       & 5 &\\ \hline
	    Le robot doit se retirer de la zone de dessin une fois celui-ci terminé 									& 3     & 1     & 1     & 5     & 5     & 5     & 5     & 5     &       &       &       &       &       &       &  	&\\ \hline
	    Respecter un budget de 250\$ 																				& 5     & 5     & 2     & 2     & 3     & 3     & 5     & 5     &       &       &       &       &       &       &  	&\\ \hline
	    Respecter l'échéancier 																						& 3     & 3     & 3     & 3     & 3     & 3     & 3     & 3     & 1     & 1     & 1     & 1     & 1     & 1     & 1 &\\ \hline
		Avoir le temps d'exécution de la tâche																		&		&		&		&		&		&		&		&		&		&		&		&		&		&		&	& 5 \\ \hline
	\end{tabular}}
	\end{minipage}}
\end{table}

\eject \pdfpagewidth=8.5in \pdfpageheight=11in
\textwidth=\hcolw