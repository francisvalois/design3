% !TEX encoding = UTF-8 Unicode

%
% Exemple de rapport
% par Pierre Tremblay, Universite Laval
% modifié par Christian Gagne, Universite Laval
% modifié par Francis Valois, Université Laval
% 31/01/2011 - version 1.4
%

%
% Modele d'organisation d'un projet LaTeX 
% rapport/      dossier racine et fichier principal
% rapport/fig   fichiers des figures
% rapport/tex   autres fichiers .tex
%

% ** Preambule **
%
% Ajouter les options au besoin :
%    - "ULlof" pour inclure la liste des figures, requis si "\begin{figure}" utilise
%    - "ULlot" pour inclure la liste des tableaux, requis si "\begin{table}" utilise
%
\documentclass[12pt,ULlof,ULlot]{ULrapport}

% Chargement des packages supplementaires (si absent de la classe)
\usepackage[utf8]{inputenc}
\usepackage[T1]{fontenc}
\usepackage[autolanguage]{numprint}
\usepackage{icomma}
\usepackage{subfigure}
\usepackage{graphicx}
\usepackage[absolute]{textpos}
\usepackage[final]{pdfpages}
\def\dbar{{\mathchar'26\mkern-12mu d}} 
%\usepackage[options]{nom_du_package}

% Definition d'une commande pour presenter des cellules multilignes dans un tableau
\newcommand{\cellulemultiligne}[1]{\begin{tabular}{@{}c@{}}#1\end{tabular}}


% Definition de colonnes en mode paragraphe avec alignement ajustable
% Cette definition requiert le chargement du package "array"
%    - alignement horizontal, parametre #1 : - \raggedright (aligne a gauche)
%                                            - \centering (centre)
%                                            - \raggedleft (aligne a droite)
%    - alignement vertical, parametre #2 : - p (aligne en haut)
%                                          - m (centre)
%                                          - b (aligne en bas)
%    - largeur, parametre #3 : longueur
\newcolumntype{Z}[3]{>{#1\hspace{0pt}\arraybackslash}#2{#3}}

% Definitions des parametres de la page titre
\TitreProjet{Livrable 1 - Robot Kinocto}                         % Titre du projet
\TitreRapport{Machines Électriques}       % Titre du rapport
\Destinataire{M. Dominique Grenier, M. Luc Lamontagne et M. Abdelhakim Bendada}         % Nom(s) du destinataire
\TableauMembres{%                                     % Tableau des membres de l'equipe
   XXX\,XXX\,XXX  & Émile Arsenault \\\hline 
   908\,190\,985  & Philippe Bourdages \\\hline
   910\,098\,468  & Pierre-Luc Buhler \\\hline    
   XXX\,XXX\,XXX & Diane Fournier \\\hline
   908\,159\,170  & Imane Mouhtij \\\hline
   908\,318\,388  & Olivier Sylvain \\\hline
   XXX\,XXX\,XXX  & Daniel Thibodeau \\\hline
   XXX\,XXX\,XXX  & Francis Valois \\\hline
}
\DateRemise{11 février 2012}                           % Date de remis


% Corps du document

\begin{document}

%   Chapitres
%!TEX root = ../rapport.tex
%!TEX encoding = UTF-8 Unicode

% Chapitres "Introduction"

% modifié par Francis Valois, Université Laval
% 31/01/2011 - version 1.0 - Création du document

\chapter{Introduction}
\label{s:introduction}

%!TEX root = ../rapport.tex
%!TEX encoding = UTF-8 Unicode

% Chapitres "Introduction"

% modifié par Francis Valois, Université Laval
% 31/01/2011 - version 1.0 - Création du document

\chapter{Diagramme de contexte}
\label{s:contexte}
\begin{figure}[htb]
\centering
\includegraphics[scale=0.38]{fig/DiagrammeDeContexteDesign3.png}
\end{figure}

%!TEX root = ../rapport.tex
%!TEX encoding = UTF-8 Unicode

% Chapitres "Introduction"

% modifié par Francis Valois, Université Laval
% 31/01/2011 - version 1.0 - Création du document
\chapter{Description des propriétés fonctionnelles}
\label{s:fonctionnelles}
Pour simplifier la lecture du tableau de la description des propriétés fonctionnelles, celui-ci a été séparé en trois pages selon trois différentes sections: vision et traitement numérique (présenté dans le tableau \ref{tab:dpf1}), communication et déplacement (présenté dans le tableau \ref{tab:dpf2}) ainsi qu'alimentation et affichage (présenté dans le tableau \ref{tab:dpf3}). 
\begin{landscape}
\begin{table}[!ht]
\centering
	\caption{Description des propriétés fonctionnelles: section "Vision et Traitement Numérique"} 
	\label{tab:dpf1}
	\small
	\scalebox{0.8}{
	\tabcolsep=0.11cm
	\begin{tabular}{|Z{\raggedright}{m}{6.5cm}||Z{\centering}{m}{2cm}|Z{\centering}{m}{2cm}|Z{\centering}{m}{2cm}|Z{\centering}{m}{2cm}|Z{\centering}{m}{2cm}|Z{\centering}{m}{2cm}|Z{\centering}{m}{2cm}|Z{\centering}{m}{2cm}|Z{\centering}{m}{2cm}|} 
		
		\cline{2-10}
		\multicolumn{1}{c}{} 																						& \multicolumn{9}{|c|}{\textbf{Fonctionnalités}} \\ \cline{2-10}
		\multicolumn{1}{c}{} 																						& \multicolumn{3}{|c}{Vision numérique} 																		& \multicolumn{6}{|c|}{Traitement numérique}																																											\\ \cline{2-10}
		\multicolumn{1}{c}{} 																						& \multicolumn{1}{|Z{\centering}{m}{2cm}|}{Détecter obstacles} 	& Localiser le robot 	& Lire le cube 			& Calculer la trajectoire optimale 	& \multicolumn{2}{Z{\centering}{m}{4cm}|}{Contrôler le robot pour le dessin} 	& Décoder le signal d'antenne 	& Choisir le cube selon le signal d'antenne & Résoudre le sudocube \\ \hline
		\centering\textbf{Exigences du client}																		& Temps de calcul (s) 											& Taux d'erreur (\%) 	& Temps de calcul (s) 	& Temps de calcul (s) 				& Temps de calcul (s) 	& Taux d'erreur (\%) 									& 								& 											& Temps de calcul (s) 	\\ \hline  \hline
		Être autonome pendant un minimum de 10 minutes 																& 3 															& 2 					& 5 					& 2 								& 3 					& 3 													& 2 							& 4 										& 2 					\\ \hline 
		Se déplacer selon la trajectoire optimale																	& 5 															& 5 					&  						& 5 								&  						&  														& 3 							& 3 										&  						\\ \hline
		Effectuer une séquence complète en moins de 10 minutes 														& 5 															& 5 					& 5 					& 5 								& 5 					&  														&  								& 3 										& 3 					\\ \hline

		Alimenter le Mac mini avec une tension de 22V à 30V et ondulation de tension inférieure à 200 mV 									&  &  &  &  &  &  &  &  &  \\ \hline
		Résoudre sudocube 																							&  &  & 5 &  &  &  &  &  & 5 \\ \hline 
		Dessiner le chiffre selon le signal d'antenne dans une zone prédéfinie (jaune) avec une précision de ± 1cm 	&  &  &  &  &  & 5 &  &  &  \\ \hline
		Éviter les obstacles																						& 5 & 5 &  &  &  &  &  &  & 5 \\ \hline 
		Analyser le bon cube selon le signal d'antenne 																&  &  &  &  &  &  &  &  &  \\ \hline 
		Utiliser la communication sans fil 																			&  &  &  &  &  &  &  &  &  \\ \hline 
		Concevoir un système de préhension pour le crayon 															&  &  &  &  &  &  &  &  &  \\ \hline 
		Afficher position réelle																					& 5 &  &  &  &  & 5 &  &  & 3 \\ \hline 
		Afficher la trajectoire optimale 																			&  &  & 5 &  &  &  &  &  &  \\ \hline 
		Afficher le cube résolu sur la base 																		&  &  &  &  &  &  &  &  &  \\ \hline 
		Allumer une DEL lorsque tâche terminée 																		&  &  &  &  &  &  &  &  &  \\ \hline 
		Afficher message de fin 																					&  &  &  &  &  &  &  &  &  \\ \hline
		Afficher message de départ 																					&  &  &  &  &  &  &  &  &  \\ \hline 
		Afficher trajectoire réelle avec un délai maximum de 15s 													&  &  &  &  &  &  &  &  &  \\ \hline 
		Afficher les informations sur le robot 																		&  &  &  &  &  &  &  &  &  \\ \hline 
		Respecter un budget de 250\$ 																				&  &  &  &  &  &  & 3 &  &  \\ \hline 
	\end{tabular}}
\end{table}

\begin{table}[!ht]
\centering
	\caption{Description des propriétés fonctionnelles: section "Communication et Déplacement"} 
	\label{tab:dpf2}
	\small
	\scalebox{0.8}{
	\tabcolsep=0.11cm
	\begin{tabular}{|Z{\raggedright}{m}{6.2cm}||Z{\centering}{m}{2cm}|Z{\centering}{m}{3cm}|Z{\centering}{m}{3cm}|Z{\centering}{m}{1.8cm}|Z{\centering}{m}{2.5cm}|Z{\centering}{m}{2cm}|Z{\centering}{m}{2cm}|Z{\centering}{m}{1.5cm}|Z{\centering}{m}{1.5cm}|} 
		
		\cline{2-10}
		\multicolumn{1}{c|}{} 																						& \multicolumn{9}{c|}{\textbf{Fonctionnalités}} \\ \cline{2-10}
		\multicolumn{1}{c|}{} 																						& \multicolumn{7}{c|}{Communication} & \multicolumn{2}{c|}{Déplacement}\\ \cline{2-10}
		\multicolumn{1}{c|}{} 																						& \multicolumn{1}{Z{\centering}{m}{2cm}|}{Recevoir le signal d'antenne} & Communiquer entre le robot et la station de base & Communiquer entre le Mac mini et le microcontrôleur & Commander les moteurs & Transmettre les images de la caméra vers le Mac & Contrôler la position de la caméra & Commander le préhenseur du crayon & \multicolumn{2}{Z{\centering}{m}{3cm}|}{Se déplacer sans toucher aux obstacles} \\ \hline
		\centering\textbf{Exigences du client}																		&  & Vitesse (Mo/s) & Vitesse (bits/s) &  &  &  & & Résolution (Degrés) & Vitesse (m/s) \\ \hline  \hline
		Être autonome pendant un minimum de 10 minutes 																& 4 &  &  &  &  &  &  & 3 & 1 \\ \hline 
		Se déplacer selon la trajectoire optimale																	& 3 &  & 3 & 4 & 2 &  &  & 5 & 5 \\ \hline
		Effectuer une séquence complète en moins de 10 minutes 														& 5 &  & 3 & 4 &  &  &  & 5 & 5 \\ \hline
		Alimenter le Mac mini avec une tension de 22V à 30V et une ondulation de tension inférieure à 200 mV 									&  &  &  &  &  &  &  &  &  \\ \hline
		Résoudre sudocube 																							&  &  &  &  & 3 & 1 &  &  &  \\ \hline 
		Dessiner le chiffre selon le signal d'antenne dans une zone prédéfinie (jaune) avec une précision de ± 1cm 	&  &  & 5 & 3 &  &  & 5 & 5 & 5 \\ \hline
		Éviter les obstacles																						&  &  & 5 & 3 & 4 &  &  & 5 &  \\ \hline 
		Analyser le bon cube selon le signal d'antenne 																& 5 &  &  &  &  &  &  &  &  \\ \hline 
		Utiliser la communication sans fil 																			&  &  &  &  &  &  &  &  &  \\ \hline 
		Concevoir un système de préhension pour le crayon 															&  &  &  &  &  &  & 5 &  &  \\ \hline 
		Afficher position réelle																					&  &  &  & 5 &  &  &  &  &  \\ \hline 
		Afficher la trajectoire optimale 																			& 3 & 5 &  &  &  &  &  &  &  \\ \hline 
		Afficher le cube résolu la base 																			&  & 5 &  &  & 3 &  &  &  &  \\ \hline 
		Allumer une DEL lorsque tâche terminée 																		&  &  & 5 &  &  &  &  &  &  \\ \hline 
		Afficher message de fin 																					&  & 5 &  & 5 &  &  &  &  &  \\ \hline
		Afficher message de départ 																					&  &  &  &  &  &  &  &  &  \\ \hline 
		Afficher trajectoire réelle avec un délai maximum de 15s 													&  &  &  &  &  &  &  &  &  \\ \hline 
		Afficher les informations sur le robot 																		&  &  &  &  &  &  &  &  &  \\ \hline 
		Respecter un budget de 250\$ 																				& 3 &  & 1 & 5 &  &  & 2 &  &  \\ \hline 
	\end{tabular}}
\end{table}

\begin{table}[!ht]
\centering
	\caption{Description des propriétés fonctionnelles: section "Alimentation et affichage"} 
	\label{tab:dpf3}
	\small
	\scalebox{0.8}{
	\tabcolsep=0.11cm
	\begin{tabular}{|Z{\raggedright}{m}{6.2cm}||Z{\centering}{m}{2cm}|Z{\centering}{m}{1.5cm}|Z{\centering}{m}{1.5cm}|Z{\centering}{m}{2.2cm}|Z{\centering}{m}{2cm}|Z{\centering}{m}{1.4cm}|Z{\centering}{m}{2.1cm}|Z{\centering}{m}{1.5cm}|Z{\centering}{m}{1.5cm}|Z{\centering}{m}{1.5cm}|Z{\centering}{m}{1.5cm}|} 
		
		\cline{2-12}
		\multicolumn{1}{c|}{} 																						& \multicolumn{11}{c|}{\textbf{Fonctionnalités}} \\ \cline{2-12}
		\multicolumn{1}{c|}{} 																						& \multicolumn{4}{c}{Alimentation} & \multicolumn{7}{|c|}{Affichage}\\ \cline{2-12}
		\multicolumn{1}{c|}{} 																						& \multicolumn{1}{Z{\centering}{m}{2cm}|}{Utiliser une pile rechargeable} & Alimenter les moteurs & Alimenter le Mac & Alimenter les différents périphériques & Afficher message d'initiation de la tâche & Afficher le cube résolu & Allumer la DEL lorsque tâche complétée & Afficher la trajectoire optimale & Afficher la position réelle & Afficher message de fin & Afficher sur le LCD\\ \hline
		\centering\textbf{Exigences du client}																		&  & Puissance (W) & & Ondulation de tension (V) &  &  & & & Temps d'actualisation (s) & & \\ \hline  \hline
		Être autonome pendant un minimum de 10 minutes 																& 5 & 3 & 3 & 3 &  &  &  &  &  &  & \\ \hline 
		Se déplacer selon la trajectoire optimale																	& 5 & 5 & 5 & 3 &  &  &  &  &  &  & \\ \hline 
		Effectuer une séquence complète en moins de 10 minutes 														&  &  &  &  &  &  &  &  &  &  & \\ \hline 
		Alimenter le Mac mini avec une tension de 22V à 30V \textbf{et} une ondulation de tension inférieure à 200 mV 									&  &  & 5 & 5 &  &  &  &  &  &  & \\ \hline 
		Résoudre sudocube 																							& 5 &  & 5 & 5 &  &  &  &  &  &  & \\ \hline 
		Dessiner le chiffre selon le signal d'antenne dans une zone prédéfinie (jaune) avec une précision de ± 1cm 	& 3 & 3 & 3 & 3 &  &  &  &  &  &  & \\ \hline 
		Éviter les obstacles																						& 3 & 3 & 3 & 3 &  &  &  &  &  &  & \\ \hline
		Analyser le bon cube selon le signal d'antenne 																& 5 &  & 1 & 3 & 3 & 3 & 3 &  &  &  & \\ \hline
		Utiliser la communication sans fil 																			 &  &  &  &  &  &  &  &  &  &  & \\ \hline 
		Concevoir un système de préhension pour le crayon 															&  &  &  &  &  &  &  &  &  &  & \\ \hline 
		Afficher position réelle																					&  &  & 1 & 3 &  &  & 3 &  &  &  & 5\\ \hline  
		Afficher la trajectoire optimale 																			& 3 &  & 1 & 3 & 3 & 3 & 3 &  &  & 5 & \\ \hline 
		Afficher le cube résolu sur la base 																		&  & 5 & 1 &  & 3 & 3 & 3 & 5 &  &  & \\ \hline 
		Allumer une DEL lorsque tâche terminée 																		&  &  & 1 &  &  &  & 2 &  & 5 &  & \\ \hline
		Afficher message de fin 																					&  &  &  &  &  &  &  &  &  &  & \\ \hline
		Afficher message de départ 																					 &  &  &  & 3 & 5 &  &  &  &  &  & \\ \hline 
		Afficher trajectoire réelle avec un délai maximum de 15s 													&  &  &  & 3 &  &  &  &  & 5 &  & \\ \hline  
		Afficher les informations sur le robot 																		&  &  &  & 3 &  &  &  &  &  &  & 5\\ \hline  
		Respecter un budget de 250\$ 																				& 5 & 2 & 3 & 5 &  &  &  &  &  &  & \\ \hline  
	\end{tabular}}
\end{table}



\end{landscape}
%!TEX root = ../rapport.tex
%!TEX encoding = UTF-8 Unicode

% Chapitres "Introduction"

% modifié par Francis Valois, Université Laval
% 31/01/2011 - version 1.0 - Création du document

\chapter{Description des cas d'utilisation}
\label{s:utilisation}
Cette section contient un résumé de chacun des différents cas d'utilisation. Elle est divisée en trois sous-sections: soit les cas d'utilisation en lien à l'usager, ceux en lien à la station de base et finalement ceux en lien au robot. Le diagramme des cas d'utilisation est présenté à la section \ref{use_cases}.
\section{Cas d'utilisation en lien avec l'usager}
\subsection{Lancer le signal de départ}
À l'aide d'un Graphical user interface ou interface graphique pour utilisateur (GUI) installé sur la station de base, l'utilisateur clique sur un bouton qui permet au robot de lancer son exécution (signal de départ transmis au robot pour commencer à effectuer une tâche).
\subsection{Entrer les coordonnées initiales du robot}
Avant d'envoyer le signal de départ au robot, l'utilisateur doit pouvoir entrer dans le GUI de la station de base les coordonnées de la position de départ du robot par rapport au terrain de jeu.
\section{Cas d'utilisations en lien avec la station de base}
\subsection{Localiser le robot}
À l'aide de la Kinect, la station de base doit être en mesure de trouver la position du robot sur le terrain de jeu.
\subsection{Détecter les obstacles}
La station de base, au moyen de la Kinect, doit pouvoir à détecter la position des deux obstacles disposés entre les deux zones principales du terrain de jeu, ainsi que les limites (murs) du terrain.
\subsection{Afficher la solution du sudocube}
La station de base doit être capable d'afficher une image du sudocube résolu et d'indiquer à l'écran le chiffre qui se trouve dans la case rouge du sudocube.
\subsection{Afficher le message de confirmation du lancement de la tâche}
La station doit être capable d'afficher un message de confirmation du lancement de la tâche pour l'usager lorsque le message de confirmation du robot est reçu.
\subsection{Afficher la trajectoire prévue et réelle du robot}
La trajectoire calculée et transmise par le robot doit être affichée à l'écran de la station de base. De plus, à l'aide des données fréquemment transmises par la Kinect contenant la position du robot, la station doit afficher, en comparaison à la trajectoire prévue, la trajectoire réelle du robot.
\subsection{Transmettre des données au robot}
La station doit transmettre au robot, par connexion sans fil, des messages indiquant de lancer une nouvelle tâche, la position des obstacles et la position de ces derniers.
\subsection{Recevoir des données provenant du robot}
La station de base doit être capable de recevoir, par connexion sans fil, des données provenant du robot comme la trajectoire que le robot prévoit emprunter, le sudocube résolu, le message de confirmation du lancement d'une tâche et le message indiquant la fin d'une tâche.
\subsection{Afficher le message confirmant que la tâche a été complétée avec succès ou a échoué}
La station de base doit afficher un message à l'écran confirmant que la tâche a été complétée avec succès ou un message d'échec si cette dernière a pris de plus de 10 minutes pour s'exécuter.
\subsection{Afficher le temps d'exécution de la tâche}
La station de base doit être en mesure d'afficher à l'écran le temps d'exécution de la tâche.
\section{Cas d'utilisation en lien avec le robot}
\subsection{Déterminer le chemin optimal entre un point A et un point B tout en évitant les obstacles}
En connaissant la configuration et la position des obstacles, le robot doit calculer la trajectoire optimale qu'il doit emprunter pour se rendre d'un point à un autre.
\subsection{Se déplacer d'un point A au point B}
Le robot doit être capable de se déplacer, sur le terrain de jeu, d'un point A vers un point B avec un contrôle automatisé des roues.
\subsection{Indiquer son adresse IP locale pour que la station de base puisse communiquer avec le robot}
Le robot doit être capable d'indiquer son adresse IP locale à la station de base pour que celle-ci établisse une connexion sans fil avec le robot.
\subsection{Extraire les informations d'un sudocube à partir d'une photo}
À l'aide d'une caméra embarquée sur le robot, il doit prendre une photo et représenter le sudocube à partir de l'image obtenue.
\subsection{Résoudre le sudocube et identifier le chiffre dans le carré rouge}
Le robot doit effectuer des traitements sur l'image du sudocube afin de le représenter et de le résoudre par la suite. Il identifie également l'emplacement de la case rouge à l'intérieur du sudocube et détermine donc le chiffre qui s'y trouve.
\subsection{Recevoir des données provenant de la station de base}
Le robot doit être capable de recevoir, par connexion sans fil avec la station de base, des messages indiquant de lancer une nouvelle tâche, la position des obstacles et sa position.
\subsection{Transmettre des données à la station de base}
Le robot doit transmettre, par connexion sans-fil, des données à la station de base comme la trajectoire que le robot prévoit emprunter, le sudocube résolu, le message de confirmation du lancement d'une tâche et le message indiquant la fin d'une tâche.
\subsection{Contrôler une DEL}
Une fois le message de fin de tâche transmis et le robot à l'extérieur de l'air de dessin, ce dernier doit allumer une DEL qui est installée sur le robot.
\subsection{Lancer de nouvelles tâches si la compétition n'est pas terminée}
Si le temps alloué pour la compétition (10 minutes) n'est pas écoulé, le robot doit commencer une nouvelle tâche après en avoir terminer la séquence d'opération.
\subsection{Se localiser}
Le robot doit être capablee en tout temps de déterminer sa position et son orientation, à l'aide de la webcam et des données reçues, ce qu'il lui permet de connaître son emplacement par rapport à l'antenne, aux murs, aux obstacles et aux sudocubes. De plus, il doit transmettre ces données à la station de base de façon régulière.
\subsection{Décoder la transmission de l'antenne}
Le robot doit se déplacer au-dessus de l'antenne et de capter le signal transmis par celle-ci. Il doit ensuite le décoder pour trouver les paramètres indiquant le sudocube ciblé ainsi que l'orientation et la taille du chiffre à dessiner dans la zone de base.
\subsection{Dessiner, en suivant les consignes, le chiffre prélevé dans le carré rouge}
Le robot doit utiliser un préhenseur afin de poser la mine du crayon sur la table, une fois dans l'air de dessin. Il doit ensuite se déplacer de façon à dessiner le chiffre trouvé dans la case rouge du sudocube, tout en suivant les dimensions exigées et les paramètres décodés. 
\subsection{Afficher des informations sur l'écran LCD}
Les paramètres trouvés lors du décodage du signal de l'antenne doivent être affichés sur un écran LCD installé sur le robot.
\section{Diagramme des cas d'utilisation}
\label{use_cases}
\addtolength{\evensidemargin}{-1in}	
\includepdf[scale= 1]{use_cases_diagram.pdf}
%!TEX root = ../rapport.tex
%!TEX encoding = UTF-8 Unicode

% Chapitres "Introduction"

% modifié par Francis Valois, Université Laval
% 31/01/2011 - version 1.0 - Création du document

\chapter{Diagrammes de séquences}
\label{s:sequences}
Ce chapitre présente les différentes figures associées au diagramme des séquences. Ce diagramme est séparé selon quatre portions relatives aux fonctions particulières du robot. La figure \ref{fig:diagSeq1Ite1} présente les liens entre l'utilisateur et la kinocto. La figure \ref{diagSeq2Ite1} présente les liens entre la kinocto et son environnement afin de déterminer sa position. La figure \ref{diagSeq3Ite1} présente les liens entre la kinocto et la station de base pour la transmission et l'affichage de la trajectoire optimale. La lecture et la résolution du sudocube  y est aussi présenté. La figure \ref{diagSeq4Ite1} présente les liens entre la kinocto et ses différents périphériques lors de la production du dessin et de la confirmation de la résolution de la tâche.
\begin{figure}[htb]
\centering
\includegraphics[scale=1]{fig/diagrammes_sequence1.pdf}
\caption{Diagramme des séquences présentant les liens entre l'utilisateur et la kinocto}
\label{fig:diagSeq1Ite1} 
\end{figure}
\begin{figure}[htb]
\includegraphics[scale=0.9]{fig/diagrammes_sequence2.pdf}
\caption{Diagramme des séquences présentant les liens entre la kinocto et son environnement afin de déterminer sa position}
\label{diagSeq2Ite1}
\end{figure}
\begin{figure}[htb]
\includegraphics[scale=0.9]{fig/diagrammes_sequence3.pdf}
\caption{Diagramme des séquences présentant les liens entre la kinocto et la station de base pour la transmission et l'affichage de la trajectoire optimale vers la zone de lecture du sudocube, ainsi que la résolution de celui-ci}
\label{diagSeq3Ite1}
\end{figure}
\begin{figure}[htb]
\includegraphics[scale=0.9]{fig/diagrammes_sequence4.pdf}
\caption{Diagramme des séquences présentant les liens entre la kinocto et ses différents périphériques lors de la production du dessin et de la confirmation de la résolution de la tâche}
\label{diagSeq4Ite1}
\end{figure}
%!TEX root = ../rapport.tex
%!TEX encoding = UTF-8 Unicode

% Chapitres "Introduction"

% modifié par Francis Valois, Université Laval
% 31/01/2011 - version 1.0 - Création du document
\chapter{Diagrammes de classes}
	\label{s:classes}
	\addtolength{\evensidemargin}{-1in}	
\includegraphics[scale= 0.855]{class_diagram_cropped.pdf}

%!TEX root = ../rapport.tex
%!TEX encoding = UTF-8 Unicode

% Chapitres "Introduction"

% modifié par Francis Valois, Université Laval
% 31/01/2011 - version 1.0 - Création du document

\chapter{Description des prototypes}
\label{s:prototypes}
\section{Préhenseur}

Comme dispositif pour le traçage des dessins par le robot, nous opterons pour la solution la plus simple et robuste qui soit. Afin d’éviter d’augmenter inutilement la charge de travail du microcontrôleur, nous avons décidé d’éviter d’utiliser des solutions complexes qui pourraient exiger la production d’un signal particulier par le microcontrôleur. Le signal de sortie que devra produire le celui-ci sera simplement un signal binaire : « 1» lorsque le crayon devra être en mode traçage et « 0 » lorsque le crayon sera soulevé et donc en mode « attente ». Pour maintenir le crayon en position « attente » durant la durée du trajet, nous utiliserons un système de maintien magnétique qui gardera le crayon en position relevé lorsqu’il recevra le signal « 1 » logique. Pour abaisser le crayon et ainsi entrer en mode traçage, nous utiliserons un ressort à faible constante de rappel qui entrera en jeu lorsque le système de maintien magnétique relâchera le crayon « 0 » logique. Il est bien important que la constante de rappel du ressort soit faible pour ne pas briser la mine lors du contact de celle-ci avec la surface de traçage. La constante de rappelle du ressort devra également être inférieur à la force exercée par le système de maintien magnétique qui ramènera le crayon en mode attente une fois le dessin terminé.

\section{ Récepteur et décodeur du signal manchester}

Pour le récepteur du signal d’antenne, nous utiliserons un « tone decoder » qui est utilisé dans de nombreuses applications qui nécessitent la reconnaissance de certaines tonalités bien précises. Comme nous avons besoins de détecter une série tonalités identique qui se suit et forme un code, ce genre de dispositif nous convient parfaitement. Le « tone decoder » recevra le signal émis par l’antenne et détectera si le la tonalité inaudible est émise ou non. Lorsque la fréquence (ou tonalité) sera reçue par le récepteur, celui-ci présentera zéro volt en sortie et à l’opposé, lorsque le récepteur ne détectera pas la fréquence ciblée, il présentera 5V à sa sortie. Il exécutera ces opérations d’une façon suffisamment rapide pour que le segment binaire reçu soit exactement le même que celui transmis, mais complètement opposé étant donné la configuration logique du « tone decoder ».  Ce message sera ensuite inversé par l’unité de traitement suivante et traité pour que le robot exécute la tâche adéquate. L’avantage de cette approche est que le « tone decoder » est un circuit intégré très robuste aux perturbations magnétiques de son environnement, il est peu énergivore et très compact.

\section{Affichage sur le robot}

Pour l'affichage sur le robot, nous utiliserons un écran à cristaux liquides 16 x 2  caractères. Cet écran est commandé par 8 lignes de données qui permettent de transmettre des caractères ASCII et 3 lignes de contrôle, RS, R/W et E. le contraste peut être contrôlé par un potentiomètre qui fourni la référence sur la broche 3. Il peut être alimenté à même le microcontrôleur, en excluant le rétroéclairage, qui n'est pas nécessaire pour distinguer les caratères. Si le rétroéclairage devait être utilisé, il devra être alimenté sur une source indépendamment, car il nécessite 150 mA.

\section{Alimentation du Mac mini} \label{s:alim_mac_mini}
En ce qui a trait au système d'alimentation du Mac mini, il est nécessaire de considérer l'efficacité énergétique de celle-ci. La tension nominale des batteries applicables au projet se situe rarement au-dessus de la plage de tension d'alimentation du Mac mini. Ce faisant, l'usage d'un régulateur de tension classique qui nécessite une tension supérieure à la tension d'alimentation du Mac mini est impossible, à moins de procéder à une augmentation du potentiel de la batterie. Cependant, même si la tension nécessaire au bon fonctionnement d'un régulateur classique était obtenue par la mise en série de cellules, le régulateur aurait un rendement dans les environs de 50\%. Il est donc à proscrire vu les contraintes de dimensionnement qui pèsent sur la batterie. Ce qui s'offre à la conception de l'alimentation est un système de type «Boost» qui effectue une amplification de la tension continue. Ce type de circuit est le pendant en tension continue du transformateur. En soi, il est possible de réaliser la fonction d'amplification au moyen de composants discrets. Cependant, l'implantation pratique demande beaucoup de temps et de ressources. Il est préférable, vu les coûts de l'électronique actuelle, d'opter pour des composants intégrés qui effectuent le même travail avec des rendements très élevés (de l'ordre du 90\% et plus). Aussi, il est très important de considérer la stabilité de la tension de sortie de l'alimentation. L'utilisation de composant discret permet de fixer l'ondulation de tension à des niveaux inférieurs aux niveaux maximaux du Mac mini ($\pm 200mV$).

\section{Alimentation de l'électronique embarquée}
En ce qui a trait au système d'alimentation de l'électronique embarquée, le niveau de tension requis est de 5V. Comme il est possible de réaliser une tension supérieure à 5V au moyen d'agencement de cellules de batterie, l'utilisation d'un régulateur doit être considérée. Pour les mêmes raisons qu'énoncées dans la section \ref{s:alim_mac_mini}, on constate que le rendement n'est pas suffisamment élevé pour une application embarquée. Comme la tension de la batterie employée risque d'être au dessus de 5V, l'utilisation d'une technologie de type «Boost» est impossible. Il existe cependant la technologie de type «Buck», qui produit exactement l'effet contraire. L'utilisation de composés discrets requiert davantage de temps. Une solution impliquant des composés intégrés doit être envisagée. Par ailleurs, les rendements de ce type d'alimentation peuvent s'élever à plus de 90\%. Les composantes comme le microcontrôleur requièrent une tension d'alimentation stable. L'emploi d'une alimentation faite à partir de composants intégrés permet de convenir à ce besoin.

\appendix
%!TEX root = ../rapport.tex
%!TEX encoding = UTF-8 Unicode
% Chapitres "Annexes"

% modifié par Francis Valois, Université Laval
% 31/01/2011 - version 1.0 - Création du document
\chapter{Annexes}
\label{s:annexes}

\begin{figure}
  \caption{Diagramme des cas d'utilisation}
  \centering
  \includegraphics[width=10cm]{fig/use_cases_diagram.pdf}
  \label{use_cases_diagram}
\end{figure}




\end{document}
% Fin du document

