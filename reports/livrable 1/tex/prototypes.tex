%!TEX root = ../rapport.tex
%!TEX encoding = UTF-8 Unicode

% Chapitres "Introduction"

% modifié par Francis Valois, Université Laval
% 31/01/2011 - version 1.0 - Création du document

\chapter{Description des prototypes}
\label{s:prototypes}
\section{Préhenseur}

Comme dispositif pour le traçage des dessins par le robot, nous opterons pour la solution la plus simple et robuste qui soit. Afin d’éviter d’augmenter inutilement la charge de travail du microcontrôleur, nous avons décidé d’éviter d’utiliser des solutions complexes qui pourraient exiger la production d’un signal particulier par le microcontrôleur. Le signal de sortie que devra produire le celui-ci sera simplement un signal binaire : « 1» lorsque le crayon devra être en mode traçage et « 0 » lorsque le crayon sera soulevé et donc en mode « attente ». Pour maintenir le crayon en position « attente » durant la durée du trajet, nous utiliserons un système de maintien magnétique qui gardera le crayon en position relevé lorsqu’il recevra le signal « 1 » logique. Pour abaisser le crayon et ainsi entrer en mode traçage, nous utiliserons un ressort à faible constante de rappel qui entrera en jeu lorsque le système de maintien magnétique relâchera le crayon « 0 » logique. Il est bien important que la constante de rappel du ressort soit faible pour ne pas briser la mine lors du contact de celle-ci avec la surface de traçage. La constante de rappelle du ressort devra également être inférieur à la force exercée par le système de maintien magnétique qui ramènera le crayon en mode attente une fois le dessin terminé.

\section{Récepteur et décodeur du signal manchester}

Pour le récepteur du signal magnétique, nous utiliserons un « tone decoder » qui est utilisé dans de nombreuses applications qui nécessitent la reconnaissance de certaines tonalités bien précises. Comme nous avons besoins de détecter une série tonalités identiques qui se suivent et forme un code, ce genre de dispositif nous convient parfaitement. Le « tone decoder » recevra le signal par une boucle de fil installées à un endroit sur le robot qui captera la variation du flux magnétique produite par le fil sous la table et détectera si la tonalité inaudible est émise ou non. Lorsque la fréquence (ou tonalité) sera reçue par le récepteur, celui-ci présentera zéro volt en sortie et à l’opposé, lorsque le récepteur ne détectera pas la fréquence ciblée, il présentera 5V à sa sortie. Il exécutera ces opérations d’une façon suffisamment rapide pour que le segment binaire reçu soit exactement le même que celui transmis, mais complètement opposé étant donné la configuration logique du « tone decoder ».  Ce message sera ensuite inversé par l’unité de traitement suivante et traité pour que le robot exécute la tâche adéquate. L’avantage de cette approche est que le « tone decoder » est un circuit intégré très robuste aux perturbations magnétiques de son environnement, il est peu énergivore et très compact.

\section{Microcontrôleur} \label{s:micro}

Le lien entre le pont en H contrôlant les moteurs, le signal décodé par le récepteur de signal, le préhenseur et le Mac mini sera un microcontrôleur modèle Texas Instrument Stellaris LM3S9B92. Ce microcontrôleur possèdes un grand nombre de broches d'entrée/sortie, 4 PWM, 2 interfaces d'encodeur à quadrature qui permettront le contrôle et l'asservissement des moteurs Il possède également deux ports de communication UART ainsi qu'une interface USB pour la communication avec le Mac mini. Le microcontrôleur peut être programmé en C avec un compilateur, un IDE et des librairies de pilotes de périphériques fournies.   

\section{Affichage sur le robot} \label{s:LCD}

Pour l'affichage sur le robot, nous utiliserons un écran à cristaux liquides 16 x 2  caractères. Cet écran est commandé par 8 lignes de données qui permettent de transmettre des caractères ASCII et 3 lignes de contrôle, RS, R/W et E. Le huitième bit de données peut également servir de "busy flag", ce qui permet de savoir si le contrôleur de l'écran a terminé d'exécuter les dernières instructions. Il est possible d'ajouter des caractères "maison" dans des espaces libres de la mémoire de caractères. Le contraste peut être contrôlé par un potentiomètre qui fourni la référence sur la broche 3. Il peut être alimenté à même le microcontrôleur, en excluant le rétroéclairage, qui n'est pas nécessaire pour distinguer les caratères dans un environnement éclairé. Si le rétroéclairage devait être utilisé, il devra être alimenté sur une source indépendante du microcontrôleur, car il nécessite 150 mA. Les fonctions codées dans les fichiers ecran.h et ecran.c de l'annexe~\ref{} permettent d'envoyer des caractères et des chaînes de caractères sur le LCD et de contrôler le déplacement du curseur.

\section{Communication entre le microcontrôleur et le Mac mini} \label{s:comm_mac_micro}

Comme le microcontrôleur possède une interface USB qui permet la communication avec l'ordinateur via un port série UART, nous utiliserons le port UART pour la communication. 

\section{Alimentation du Mac mini} \label{s:alim_mac_mini}
En ce qui a trait au système d'alimentation du Mac mini, le dispostif à concevoir doir réussir à élever la tension de la batterie jusqu'à 24V. Cette tension correspond à un point de fonctionnement qui est jugé comme idéal selon les spécifications techniques de l'appareil. On constate par la suite que la puissance demandée par le Mac mini sera la plus grande dépense énergétique du système. Il faut donc envisager une alimentation avec un très bon rendement, de manière à limiter le dimensionnement de la batterie. L'usage d'un régulateur de tension conventionnel, qui utilise une tension supérieure à l'alimentation, n'est pas envisageable pour le projet, dans la mesure ou le rendement dépasse rarement les 50\%. Ce qui s'offre à la conception de l'alimentation est un système de type «Boost» qui effectue une amplification de la tension continue. Ce type de circuit est le pendant en tension continue du transformateur. En soi, il est possible de réaliser la fonction d'amplification au moyen de composants discrets. Cependant, l'implantation pratique demande beaucoup de temps et de ressources et est généralement moins robuste qu'une alimentation utilisant des composants intégrés. Il est préférable, vu les coûts de l'électronique actuelle, d'opter pour des composants intégrés qui effectuent le même travail avec des rendements très élevés (de l'ordre du 90\% et plus). Aussi, vu le coût de l'ordinateur et sa sensibilité aux oscillations de tension, il est très important de considérer la stabilité de la tension de sortie de l'alimentation. Les régulateurs envisageables pour l'application et les spécifications en courant présentent une ondulation de tension inférieure aux niveaux maximaux du Mac mini ($\pm 200mV$). Il est donc tout indiqué d'opter pour un circuit intégré de type «Boost», vu son bon rendement, sa robustesse, sa facilité d'implantation ainsi que sa stabilité de tension de sortie.

\section{Alimentation de l'électronique embarquée}
En ce qui a trait au système d'alimentation de l'électronique embarquée, l'alimentation de l'électronique autre que le Mac mini doit se faire optimalement en 5V, puisqu'il s'agit d'une tension utilisable pour le servomoteur de la tourelle de caméra et permet aussi de donner un point de référence au pont en H. Par ailleurs, la plupart des microcontrôleurs ont des entrées 5V, il est donc de mise d'utiliser cette tension pour l'électronique auxiliaire. Le dispositif devra abaisser la tension de la batterie à 5V puisque celle-ci sera sélectionnée avec une tension supérieure. On peut tout de suite songer à l'usage d'un régulateur, cependant, pour les mêmes raisons qu'énoncées à la section \ref{s:alim_mac_mini}, un faible rendement de l'alimentation pourrait dégrader la durée de vie de la batterie. Vu qu'il est nécessaire d'abaisser la tension, l'utilisation d'une technologie de type «Boost» est impossible. Il existe cependant la technologie de type «Buck», qui produit exactement l'effet contraire. L'utilisation de composés discrets requiert davantage de temps et est moins robuste. Une solution impliquant des composés intégrés doit être envisagée. Par ailleurs, les rendements de ce type d'alimentation peuvent s'élever à plus de 90\%. Les composantes comme le microcontrôleur et le pont en H requièrent une tension d'alimentation stable et l'emploi d'une alimentation faite à partir de composants intégrés permet de convenir à ce besoin. Il est donc tout indiqué d'opter pour un circuit intégré de type «Buck», vu son bon rendement, sa robustesse, sa facilité d'implantation ainsi que sa stabilité de tension de sortie.