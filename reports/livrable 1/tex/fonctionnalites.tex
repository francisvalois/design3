%!TEX root = ../rapport.tex
%!TEX encoding = UTF-8 Unicode

% Chapitres "Introduction"

% modifié par Francis Valois, Université Laval
% 31/01/2011 - version 1.0 - Création du document
\chapter{Description des propriétés fonctionnelles}
\label{s:fonctionnelles}
Pour simplifier la lecture du tableau de la description des propriétés fonctionnelles, celui-ci a été séparé en trois pages selon trois différentes sections: vision et traitement numérique (présenté dans le tableau \ref{tab:dpf1}), communication et déplacement (présenté dans le tableau \ref{tab:dpf2}) ainsi qu'alimentation et affichage (présenté dans le tableau \ref{tab:dpf3}). 
\begin{landscape}
\begin{table}[!ht]
\centering
	\caption{Description des propriétés fonctionnelles: section "Vision et Traitement Numérique"} 
	\label{tab:dpf1}
	\small
	\scalebox{0.8}{
	\tabcolsep=0.11cm
	\begin{tabular}{|Z{\raggedright}{m}{6.5cm}||Z{\centering}{m}{2cm}|Z{\centering}{m}{2cm}|Z{\centering}{m}{2cm}|Z{\centering}{m}{2cm}|Z{\centering}{m}{2cm}|Z{\centering}{m}{2cm}|Z{\centering}{m}{2cm}|Z{\centering}{m}{2cm}|Z{\centering}{m}{2cm}|} 
		
		\cline{2-10}
		\multicolumn{1}{c}{} 																						& \multicolumn{9}{|c|}{\textbf{Fonctionnalités}} \\ \cline{2-10}
		\multicolumn{1}{c}{} 																						& \multicolumn{3}{|c}{Vision numérique} 																		& \multicolumn{6}{|c|}{Traitement numérique}																																											\\ \cline{2-10}
		\multicolumn{1}{c}{} 																						& \multicolumn{1}{|Z{\centering}{m}{2cm}|}{Détecter obstacles} 	& Localiser le robot 	& Lire le cube 			& Calculer la trajectoire optimale 	& \multicolumn{2}{Z{\centering}{m}{4cm}|}{Contrôler le robot pour le dessin} 	& Décoder le signal d'antenne 	& Choisir le cube selon le signal d'antenne & Résoudre le sudocube \\ \hline
		\centering\textbf{Exigences du client}																		& Temps de calcul (s) 											& Taux d'erreur (\%) 	& Temps de calcul (s) 	& Temps de calcul (s) 				& Temps de calcul (s) 	& Taux d'erreur (\%) 									& 								& 											& Temps de calcul (s) 	\\ \hline  \hline
		Être autonome pendant un minimum de 10 minutes 																& 3 															& 2 					& 5 					& 2 								& 3 					& 3 													& 2 							& 4 										& 2 					\\ \hline 
		Se déplacer selon la trajectoire optimale																	& 5 															& 5 					&  						& 5 								&  						&  														& 3 							& 3 										&  						\\ \hline
		Effectuer une séquence complète en moins de 10 minutes 														& 5 															& 5 					& 5 					& 5 								& 5 					&  														&  								& 3 										& 3 					\\ \hline

		Alimenter le Mac mini avec une tension de 22V à 30V et ondulation de tension inférieure à 200 mV 									&  &  &  &  &  &  &  &  &  \\ \hline
		Résoudre sudocube 																							&  &  & 5 &  &  &  &  &  & 5 \\ \hline 
		Dessiner le chiffre selon le signal d'antenne dans une zone prédéfinie (jaune) avec une précision de ± 1cm 	&  &  &  &  &  & 5 &  &  &  \\ \hline
		Éviter les obstacles																						& 5 & 5 &  &  &  &  &  &  & 5 \\ \hline 
		Analyser le bon cube selon le signal d'antenne 																&  &  &  &  &  &  &  &  &  \\ \hline 
		Utiliser la communication sans fil 																			&  &  &  &  &  &  &  &  &  \\ \hline 
		Concevoir un système de préhension pour le crayon 															&  &  &  &  &  &  &  &  &  \\ \hline 
		Afficher position réelle																					& 5 &  &  &  &  & 5 &  &  & 3 \\ \hline 
		Afficher la trajectoire optimale 																			&  &  & 5 &  &  &  &  &  &  \\ \hline 
		Afficher le cube résolu sur la base 																		&  &  &  &  &  &  &  &  &  \\ \hline 
		Allumer une DEL lorsque tâche terminée 																		&  &  &  &  &  &  &  &  &  \\ \hline 
		Afficher message de fin 																					&  &  &  &  &  &  &  &  &  \\ \hline
		Afficher message de départ 																					&  &  &  &  &  &  &  &  &  \\ \hline 
		Afficher trajectoire réelle avec un délai maximum de 15s 													&  &  &  &  &  &  &  &  &  \\ \hline 
		Afficher les informations sur le robot 																		&  &  &  &  &  &  &  &  &  \\ \hline 
		Respecter un budget de 250\$ 																				&  &  &  &  &  &  & 3 &  &  \\ \hline 
	\end{tabular}}
\end{table}

\begin{table}[!ht]
\centering
	\caption{Description des propriétés fonctionnelles: section "Communication et Déplacement"} 
	\label{tab:dpf2}
	\small
	\scalebox{0.8}{
	\tabcolsep=0.11cm
	\begin{tabular}{|Z{\raggedright}{m}{6.2cm}||Z{\centering}{m}{2cm}|Z{\centering}{m}{3cm}|Z{\centering}{m}{3cm}|Z{\centering}{m}{1.8cm}|Z{\centering}{m}{2.5cm}|Z{\centering}{m}{2cm}|Z{\centering}{m}{2cm}|Z{\centering}{m}{1.5cm}|Z{\centering}{m}{1.5cm}|} 
		
		\cline{2-10}
		\multicolumn{1}{c|}{} 																						& \multicolumn{9}{c|}{\textbf{Fonctionnalités}} \\ \cline{2-10}
		\multicolumn{1}{c|}{} 																						& \multicolumn{7}{c|}{Communication} & \multicolumn{2}{c|}{Déplacement}\\ \cline{2-10}
		\multicolumn{1}{c|}{} 																						& \multicolumn{1}{Z{\centering}{m}{2cm}|}{Recevoir le signal d'antenne} & Communiquer entre le robot et la station de base & Communiquer entre le Mac mini et le microcontrôleur & Commander les moteurs & Transmettre les images de la caméra vers le Mac & Contrôler la position de la caméra & Commander le préhenseur du crayon & \multicolumn{2}{Z{\centering}{m}{3cm}|}{Se déplacer sans toucher aux obstacles} \\ \hline
		\centering\textbf{Exigences du client}																		&  & Vitesse (Mo/s) & Vitesse (bits/s) &  &  &  & & Résolution (Degrés) & Vitesse (m/s) \\ \hline  \hline
		Être autonome pendant un minimum de 10 minutes 																& 4 &  &  &  &  &  &  & 3 & 1 \\ \hline 
		Se déplacer selon la trajectoire optimale																	& 3 &  & 3 & 4 & 2 &  &  & 5 & 5 \\ \hline
		Effectuer une séquence complète en moins de 10 minutes 														& 5 &  & 3 & 4 &  &  &  & 5 & 5 \\ \hline
		Alimenter le Mac mini avec une tension de 22V à 30V et une ondulation de tension inférieure à 200 mV 									&  &  &  &  &  &  &  &  &  \\ \hline
		Résoudre sudocube 																							&  &  &  &  & 3 & 1 &  &  &  \\ \hline 
		Dessiner le chiffre selon le signal d'antenne dans une zone prédéfinie (jaune) avec une précision de ± 1cm 	&  &  & 5 & 3 &  &  & 5 & 5 & 5 \\ \hline
		Éviter les obstacles																						&  &  & 5 & 3 & 4 &  &  & 5 &  \\ \hline 
		Analyser le bon cube selon le signal d'antenne 																& 5 &  &  &  &  &  &  &  &  \\ \hline 
		Utiliser la communication sans fil 																			&  &  &  &  &  &  &  &  &  \\ \hline 
		Concevoir un système de préhension pour le crayon 															&  &  &  &  &  &  & 5 &  &  \\ \hline 
		Afficher position réelle																					&  &  &  & 5 &  &  &  &  &  \\ \hline 
		Afficher la trajectoire optimale 																			& 3 & 5 &  &  &  &  &  &  &  \\ \hline 
		Afficher le cube résolu la base 																			&  & 5 &  &  & 3 &  &  &  &  \\ \hline 
		Allumer une DEL lorsque tâche terminée 																		&  &  & 5 &  &  &  &  &  &  \\ \hline 
		Afficher message de fin 																					&  & 5 &  & 5 &  &  &  &  &  \\ \hline
		Afficher message de départ 																					&  &  &  &  &  &  &  &  &  \\ \hline 
		Afficher trajectoire réelle avec un délai maximum de 15s 													&  &  &  &  &  &  &  &  &  \\ \hline 
		Afficher les informations sur le robot 																		&  &  &  &  &  &  &  &  &  \\ \hline 
		Respecter un budget de 250\$ 																				& 3 &  & 1 & 5 &  &  & 2 &  &  \\ \hline 
	\end{tabular}}
\end{table}

\begin{table}[!ht]
\centering
	\caption{Description des propriétés fonctionnelles: section "Alimentation et affichage"} 
	\label{tab:dpf3}
	\small
	\scalebox{0.8}{
	\tabcolsep=0.11cm
	\begin{tabular}{|Z{\raggedright}{m}{6.2cm}||Z{\centering}{m}{2cm}|Z{\centering}{m}{1.5cm}|Z{\centering}{m}{1.5cm}|Z{\centering}{m}{2.2cm}|Z{\centering}{m}{2cm}|Z{\centering}{m}{1.4cm}|Z{\centering}{m}{2.1cm}|Z{\centering}{m}{1.5cm}|Z{\centering}{m}{1.5cm}|Z{\centering}{m}{1.5cm}|Z{\centering}{m}{1.5cm}|} 
		
		\cline{2-12}
		\multicolumn{1}{c|}{} 																						& \multicolumn{11}{c|}{\textbf{Fonctionnalités}} \\ \cline{2-12}
		\multicolumn{1}{c|}{} 																						& \multicolumn{4}{c}{Alimentation} & \multicolumn{7}{|c|}{Affichage}\\ \cline{2-12}
		\multicolumn{1}{c|}{} 																						& \multicolumn{1}{Z{\centering}{m}{2cm}|}{Utiliser une pile rechargeable} & Alimenter les moteurs & Alimenter le Mac & Alimenter les différents périphériques & Afficher message d'initiation de la tâche & Afficher le cube résolu & Allumer la DEL lorsque tâche complétée & Afficher la trajectoire optimale & Afficher la position réelle & Afficher message de fin & Afficher sur le LCD\\ \hline
		\centering\textbf{Exigences du client}																		&  & Puissance (W) & & Ondulation de tension (V) &  &  & & & Temps d'actualisation (s) & & \\ \hline  \hline
		Être autonome pendant un minimum de 10 minutes 																& 5 & 3 & 3 & 3 &  &  &  &  &  &  & \\ \hline 
		Se déplacer selon la trajectoire optimale																	& 5 & 5 & 5 & 3 &  &  &  &  &  &  & \\ \hline 
		Effectuer une séquence complète en moins de 10 minutes 														&  &  &  &  &  &  &  &  &  &  & \\ \hline 
		Alimenter le Mac mini avec une tension de 22V à 30V \textbf{et} une ondulation de tension inférieure à 200 mV 									&  &  & 5 & 5 &  &  &  &  &  &  & \\ \hline 
		Résoudre sudocube 																							& 5 &  & 5 & 5 &  &  &  &  &  &  & \\ \hline 
		Dessiner le chiffre selon le signal d'antenne dans une zone prédéfinie (jaune) avec une précision de ± 1cm 	& 3 & 3 & 3 & 3 &  &  &  &  &  &  & \\ \hline 
		Éviter les obstacles																						& 3 & 3 & 3 & 3 &  &  &  &  &  &  & \\ \hline
		Analyser le bon cube selon le signal d'antenne 																& 5 &  & 1 & 3 & 3 & 3 & 3 &  &  &  & \\ \hline
		Utiliser la communication sans fil 																			 &  &  &  &  &  &  &  &  &  &  & \\ \hline 
		Concevoir un système de préhension pour le crayon 															&  &  &  &  &  &  &  &  &  &  & \\ \hline 
		Afficher position réelle																					&  &  & 1 & 3 &  &  & 3 &  &  &  & 5\\ \hline  
		Afficher la trajectoire optimale 																			& 3 &  & 1 & 3 & 3 & 3 & 3 &  &  & 5 & \\ \hline 
		Afficher le cube résolu sur la base 																		&  & 5 & 1 &  & 3 & 3 & 3 & 5 &  &  & \\ \hline 
		Allumer une DEL lorsque tâche terminée 																		&  &  & 1 &  &  &  & 2 &  & 5 &  & \\ \hline
		Afficher message de fin 																					&  &  &  &  &  &  &  &  &  &  & \\ \hline
		Afficher message de départ 																					 &  &  &  & 3 & 5 &  &  &  &  &  & \\ \hline 
		Afficher trajectoire réelle avec un délai maximum de 15s 													&  &  &  & 3 &  &  &  &  & 5 &  & \\ \hline  
		Afficher les informations sur le robot 																		&  &  &  & 3 &  &  &  &  &  &  & 5\\ \hline  
		Respecter un budget de 250\$ 																				& 5 & 2 & 3 & 5 &  &  &  &  &  &  & \\ \hline  
	\end{tabular}}
\end{table}



\end{landscape}